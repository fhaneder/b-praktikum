\documentclass[ngerman]{scrbook}

\usepackage[utf8]{inputenc}
\usepackage[T1]{fontenc}
\usepackage{babel}
\usepackage{lmodern}
\usepackage{microtype}
\usepackage[colorlinks=true,linkcolor=blue]{hyperref}
\usepackage{makeidx}
\usepackage{enumitem}
\usepackage{graphicx}
\usepackage{subfigure}
\usepackage{caption}

\usepackage{amsmath}
\usepackage{amssymb}
\usepackage[version=4]{mhchem}
\usepackage[exponent-product=\cdot,per-mode=fraction]{siunitx}

\usepackage{tabularx}
\usepackage{tabulary}
\usepackage{booktabs}

\newcommand{\planck}{h}
\newcommand{\sol}{c}
\newcommand{\boltz}{k}
\newcommand{\G}{G}
\newcommand{\IN}{\mathbb{N}}
\newcommand{\IZ}{\mathbb{Z}}
\renewcommand{\d}{\mathrm{d}}
\newcommand*{\e}[1]{e^{#1}}
\newcommand{\numberthis}{\addtocounter{equation}{1}\tag{\theequation}}
\newcommand*{\pdiff}[2]{\frac{\partial#1}{\partial#2}}
\newcommand*{\pddiff}[2]{\frac{\partial^2#1}{\partial#2^2}}
\newcommand{\divg}[1]{\nabla\cdot\vec{#1}}
\newcommand{\rot}[1]{\nabla\times\vec{#1}}
\newcommand{\Lap}{\Delta}
\newcommand*{\avg}[1]{<#1>}
\newcommand*{\ten}[1]{\overline{\overline{#1}}}
\newcommand*{\skp}[2]{\vec{#1}\cdot\vec{#2}}
\newcommand*{\vecp}[2]{\vec{#1}\times\vec{#2}}
\newcommand*{\skpud}[6]{\vec{#1}^{#2}_{#3}\cdot\vec{#4}^{#5}_{#6}}
\newcommand*{\vecpud}[6]{\vec{#1}^{#2}_{#3}\times\vec{#4}^{#5}_{#6}}
\newcommand*{\uvec}[2][e]{\hat{#1}_{#2}}

\begin{document}
\title{B-Praktikum}
\author{Fabian Haneder}
\date{09.11.2016}
\maketitle[1]

\tableofcontents
\chapter{Optische Spektroskopie}
Im Jahr 1835 behauptete einst der französische Philosoph Auguste Comte, man würde nie etwas über die chemische Zusammensetzung der Sonne und der Sterne erfahren. Bereits 33 Jahre später entdeckten Sir Norman Loycker und Pierre Janssen ein bislang unbekanntes Element in der Sonne, das sie Helium nannten. Möglich war ihnen dies durch den Fortschritt der Spektroskopie, in die in diesem Versuch eingeführt werden soll.

Unter Spektroskopie versteht man physikalische Methoden, die dazu benutzt werden, elektromagnetische Strahlung und, gerade zu Anfang, vor allem sichtbares Licht nach einer bestimmten Eigenschaft wie der Wellenlänge zu zerlegen. Außer zur Untersuchung von Himmelskörpern können diese Methoden auch verwendet werden, um mittels Spektralanalyse oder Massenspektrometrie die chemische Zusammensetzung von unbekannten Proben sehr genau zu bestimmen. 

Aber nicht nur die Chemie, sondern auch die Physik hat durch Spektroskopie signifikante Fortschritte gemacht. So findet die Entwicklung der Atomphysik und  Quantenmechanik ihren Ausgangspunkt in der Beobachtung von Linienspektren verschiedener Atome und Moleküle, und auch Naturgesetze und -konstanten konnten durch spektroskopische Methoden untersucht werden.

In diesem Versuch soll, nachdem sich zunächst mit den zur Durchführung notwendigen Programmen und Analysemethoden vertraut gemacht wurde, das Sonnenspektrum ausgewertet werden. Zudem soll das für die Strahlung einer Glimmlampe verantwortliche Element identifiziert und schließlich die sogenannte additive Farbmischung untersucht werden.

\section{Vorbereitung}
\begin{enumerate}
	\item Wozu wird das Prinzip der Beugung in einem Spektrometer benötigt?
		\subitem Unterschiedliche Wellenlängen werden unterschiedlich stark gebeugt bzw. gebrochen, wodurch die Intensitätsmaxima unterschiedlicher Farben an unterschiedlichen Orten auftreten.
	\item Welche Vorteile hat ein Beugunsgitter im Vergleich zu einem Doppelspalt? Ändert sich die Lage der Maxima mit zunehmender Spaltanzahl?
		\subitem Bei einem Beugunsgitter lässt sich eine erheblich höhere Genauigkeit erzielen, da die Intensitätsmaxima deutlich schmaler und höher sind. So berechnet sich beispielsweiße die Fußpunktbreite wie folgt: $\Delta\beta=\lambda/N\cdot d$
		
		Hierbei ist $N$ die Spaltanzahl und $d$ der Spaltabstand. Mehr Spalten bedeutet also höhere Genauigkeit. Die Lage der Maxima ändert sich allerdings nicht.
	\item Beschreiben Sie das Spektrum eines Temperaturstrahlers und einer Gasentladungslampe.
		\subitem Temperaturstrahler strahlen stets ein kontinuierliches Spektrum ab, dessen Intensitätsmaximum bei Raumtemperatur im Normalfall außerhalb des sichtbaren Bereichs liegt. Wird die Temperatur erhöht, so verschiebt sich das Maximum in den sichtbaren Bereich und dort nach und nach von Rot über Gelb bis hin zu Blau.
		
		 Gasentladungslampen hingegen erzeugen ein Spektrum, bei dem die Maxima deutlich ausgeprägter auftreten, als beim kontinuierlichen Spektrum des Temperaturstrahlers. So kann zum Beispiel mit Natriumdampf-Lampen nahezu monochromatisches Licht erzeugt werden, lediglich die Bewegung der Natriumatome im heißen Dampf führt über den Dopplereffekt zur geringfügigen Verbreiterung der der Bandbreite des emittierten Lichts.
	\item Wie können mit Hilfe eines Spektrometers die chemischen Elemente der Erd- und Sonnenatmosphäre bestimmt werden? Wie können diese der jeweiligen Atmosphäre zugeordnet werden?
		\subitem Die Elemente der Erd- und Sonnenatmosphäre können durch einen Vergleich des Spektrums des Sonnenlichts mit bereits bekannten Emissionslinien der einzelnen Elemente bestimmt werden.
		Eine Zuordnung könnte etwa dadurch erfolgen, dass versucht wird, einen der Faktoren zu eliminieren, indem z.B. von einem Satelliten aufgefangenes Licht analysiert wird.
	\item Welche Temperatur (in Kelvin) besitzt die Sonnenoberfläche? Skizzieren Sie den Intensitätsverlauf eines schwarzen Körpers mit dieser Temperatur und kennzeichnen Sie die Strahldichte des sichtbaren Bereichs.
		\subitem \begin{figure}[!hbt]
			\centering
			\hypertarget{Abb1}{}
			\includegraphics{sunspectrum}
			\caption{Intensitätsspektrum eines sonnenähnlichen Schwarzkörpers}
			\label{fig:Abb1}
		\end{figure}
		Hierzu benötigt man die Formeln für die spektrale Strahldichte und das Wien'sche Verschiebungsgesetz:
		\begin{align}
			L_{S,\lambda}(\lambda,T)&=\frac{\mathrm{c}_1}{\pi\lambda^{5}}\Big[\exp\Big(\frac{\mathrm{c}_2}{\lambda T}\Big)-1\Big]^{-1}\label{1}\\
			\lambda_{\mathrm{max}}T&=\SI{2.8978e-3}{\meter\kelvin}\label{2}
		\end{align}
		mit den Konstanten 
		\begin{align*}
			\mathrm{c}_1&=2\pi\planck\sol_0^2=\SI{3.7418e-16}{\watt\meter\squared}\\
			\mathrm{c}_2&=\planck\sol_0\boltz^{-1}=\SI{1.4388e-2}{\meter\kelvin}
		\end{align*}
		Die Sonnenoberfläche besitzt eine Temperatur von $T=\SI{5777}{\kelvin}$. Damit erreicht die Intensitätsverteilung ihr Maximum bei $\lambda_{\mathrm{max}}=\SI{5.016e-7}{\meter}$. Dies entspricht der Farbe Hellgrün. (Für Abbildung s. \hyperlink{Abb1}{hier})
		
	\item Wie hoch müsste die Temperatur eines schwarzen Körpers sein, damit das Intensitätsmaximum in der Mitte des sichbaren Bereichs liegt? Warum können solche Temperaturen nicht mit einem herkömmlichen Glühdraht erreicht werden?
		\subitem Nach dem Wien'schen Verschiebungsgesetz ergibt sich mit $\lambda_{\mathrm{avg}}=\SI{550}{\nano\meter}$:
		\begin{displaymath}
			T_{\mathrm{avg}}=\SI{5268}{\kelvin}
		\end{displaymath}
		Solche Temperaturen können mit herkömmlichen Glührähten aus Wolfram nicht erreicht werden, da Wolfram einen Schmelzpunkt von $T_{\mathrm{schmelz}}=\SI{3695}{\kelvin}$  aufweist.
	\item Wie kann die Temperatur eines Glühdrahtes relativ einfach (in guter Näherung) bestimmt werden, wenn der Verlauf des Spektrums der Strahlung bekannt ist? Wie kann diese bestimmt werden, wenn nur ein kleiner Ausschnitt des Verlaufs (z.B. nur der sichtbare Bereich) bekannt ist?
		\subitem Ist das gesamte Spektrum bekannt, so kann durch Ermittlung der Wellenlänge, bei der das Intensitätsmaximum auftritt, die Temperatur des Strahlers über das Wien'sche Verschiebungsgesetz recht einfach berechnen.
		Ist nur ein Teil des Spektrums bekannt, so kann aus dem Auftreten bestimmter Spektrallinien zumindest eine Mindesttemperatur extrapoliert werden, die nötig ist, um das Material des Glühdrahtes mit der entsprechenden Wellenlänge zum Leuchten zu bringen.
	\item Erklären Sie den Begriff der Farbtemperatur. Was sagt dieser Begriff über das Spektrum einer Glühlampe(Temperaturstrahler) und einer Energiesparlampe(Gasentladungslampe) aus?
		\subitem Unter Farbtemperatur versteht man die zu einer bestimmten Farbe gehörende Temperatur, die nötig ist, um einen schwarzen Körper unter festgelegter Helligkeit und  Beobachtungsbedingungen möglichst genau mit dieser Farbe zum Strahlen zu bringen. 
		
		Die Farbtemperatur von Energiesparlampen und anderen Gasentladungslampen ist im Normalfall bei ca. $\SI{4000}{\kelvin}-\SI{5000}{\kelvin}$ und damit höher, als bei Glühlampen (ca. $\SI{2600}{\kelvin}-\SI{3000}{\kelvin}$). Eine Ausnahme bilden hier lediglich Natriumdampflampen, deren Farbtemperatur etwa $\SI{2000}{\kelvin}$ beträgt. Damit ist im Allgemeinen eine höhere Temperatur nötig, um mit einem schwarzen Körper das Leuchten einer Energiesparlampe nachzuahmen, als das einer Glühbirne.
	\item Welcher proportionale Zusammenhang besteht zwischen spezifischer Ausstrahlung und Temperatur eines schwarzen Körpers? Welches Gesetz beschreibt diesen Zusammenhang?
		\subitem Wie oben bereits beschrieben, hängen die spezifische Ausstrahlung und Temperatur eines schwarzen Körpers über das Wien'sche Gesetz \eqref{2} zusammen.
	\item Wie kann mit Hilfe eines Spektrometers auf die chemischen Inhaltsstoffe von Gasentladungslampen geschlossen werden?
		\subitem Die Spektra von Gasentladungslampen haben meist sehr ausgeprägte Maxima, man spricht auch von Linienspektra. Vergleicht man die Wellenlängen dieser Peaks mit den Emissionsspektra von bekannten Stoffen, so lässt sich auf die chemische Zusammensetzung des jeweiligen Leuchtstoffs schließen.
\end{enumerate}
\section{Durchführung}
\subsection{Allgemeine Hinweise}
Zum Wechseln der Lampen (230 Volt!)
\begin{itemize}
	\item Lampe vor Wechseln ausschalten und abkühlen lassen
	\item Nicht die Kontakte in der Fassung berühren
\end{itemize}
Zum Umgang mit dem Lichtwellenleiter
\begin{itemize}
	\item Zu starke Krümmung vermeiden ($r\geq\SI{15}{\centi\meter}$)
	\item Unnötige Verspannungen des Knickschutzes vermeiden
	\item Berührungen mit dem Lichtwellenleitereingang vermeiden, diesen nach Versuchsende verschließen
	\item Abstand zwischen Lampe und Leiter min. $\SI{15}{\centi\meter}$
\end{itemize}
Zur Versuchsdurchführung
\begin{itemize}
	\item Auf möglichst wenig Umgebungslicht achten
	\item Dunkel- und Referenzspektrum aktuell halten
	\item Richtige Parameter beim Speichern angeben
	\item Diagramme mit allen relevanten Angaben beschriften
\end{itemize}
\subsection{Einführende Versuche}
\subsubsection{Einführung in SpectraWiz}
Öffnen Sie SpectraWiz und wechseln Sie ggf. in den Scope-Modus. Setze die Parameter folgendermaßen:
\begin{displaymath}
	\mathrm{\textsc{SCOPE -> ... Time: 100 ms, Avg:1, Sm:0, Sg:0, Tc:off, Xt:3, Ch:1}}
\end{displaymath}
Richten Sie den Lichtwellenleiter auf die Glühlampe aus und fixieren Sie ihn mithilfe des Stativmaterials. Speichern Sie bei abgeschalteter Lampe das Dunkelspektrum.
\begin{itemize}[label=$\blacktriangleright$]
	\item Wozu wird das Dunkelspektrum benötigt?
\end{itemize}
Justieren Sie den Lichtwellenleiter so, dass das Maximum des Graphen im obersten Viertel der Skala liegt.
\begin{itemize}[label=$\blacktriangleright$]
	\item Wie wirkt sich eine Änderung der Parameter "\textsc{Detector integration time}" und "\textsc{Number of scans to average}" auf die Anzeige aus?
\end{itemize}
Nehmen Sie das Spektrum der Glühlampe auf und speichern Sie es ab. 
\begin{itemize}[label=$\blacktriangleright$]
	\item Begründen Sie, warum die angezeigte Intensitätsverteilung nicht der tatsächlichen Verteilung des Glühlampenspektrums entsprechen kann.
\end{itemize}
\subsubsection{Aufnahme von Transmissionsspektren}
Nehmen Sie das Glülammpenspektrum als Referenzspektrum auf und wechseln Sie in den Transmissionsmodus
\begin{itemize}[label=$\blacktriangleright$]
	\item Wozu wird das Referenzsprektrum benötigt?
	\item Woran lässt ich im Transmissionsmodus erkennen, ob das Referenzspektrum und das Dunkelspektrum richtig eingestellt, bzw. ob die beiden noch aktuell sind?
	\item Warum ist im Wellenlängenbereich unterhalb von ca. $\SI{350}{\nano\meter}$ keine sinnvolle Transmissionsmessung möglich?
\end{itemize}
Halten Sie farbige Brillengläser in den Strahlengang.
\begin{itemize}[label=$\blacktriangleright$]
	\item Beschreiben Sie qualitativ Ihre Beobachtungen bzgl. der Farbe und des dazugehörigen Transmissionsspektrums.
\end{itemize}
Nehmen Sie die Transmissionsspektra zweier einzelner Gläser und dasd Spektrum der Kombination auf und speichern Sie sie ab. Stellen Sie die Spektra in QtiPlot in einem Diagramm dar und drucken Sie dieses aus.
\begin{itemize}[label=$\blacktriangleright$]
	\item Welcher mathematische Zusammenhang gilt für die einzelnen eben erwähnten Transmissionsgrade?
\end{itemize}
(\#) Ermitteln Sie in QtiPlot rechnerisch das Transmissionsspektrum der Filterkombination. Stellen Sie dieses zusammen mit den anderen Spektren in einem Diagramm dar.
\subsubsection{Aufnahme von Emissionsspektren}
Nehmen Sie im Scope-Modus das Spektrum einer Energiesparlampe auf und speichern Sie dieses ab.
\begin{itemize}[label=$\blacktriangleright$]
	\item Beschreiben Sie qualitativ die Unterschiede zum Spektrum einer Glühlampe.
\end{itemize}
Nehmen Sie die Spektren der LEDs (rot, grün, blau) auf.
\begin{itemize}[label=$\blacktriangleright$]
	\item Notieren Sie sich zu den einzelnen Farben der LEDs die Wellenlängenwerte der Maxima, welche Sie in SpectraWiz bestimmen können.
	\item Lässt sich durch diese Werte auf den jeweiligen Farbeindruck schließen? Begründen Sie Ihre Aussage.
\end{itemize}
\subsubsection{Bestimmung der Farbtemperatur einer Glühlampe}
Öffnen Sie in QtiPlot die Umrechnungstabelle und importieren Sie das Glühlampenspektrum. Führen Sie die für die Umrechnung erforderlichen Schritte aus und stellen Sie das angepasste Spektrum dar.
\begin{itemize}[label=$\blacktriangleright$]
	\item Warum müssen zur Bestimmung der Farbtemperatur die Messdaten umgerechnet werden?
	\item Begründen Sie qualitativ, wie und wie stark sich die angelegte Wechselspannung ($f=\SI{50}{\hertz}$) im Vergleich zu einer entsprechenden Gleichspannung auf die Farbtemperatur auswirkt.
\end{itemize}
Erstellen Sie mittels des Fit-Assistenten eine geeignete Kurve für den Wellenlängenbereich von ca. 1-2000nm. Drucken Sie das Diagramm, in dem die ermittelte Temperatur ersichtlich ist, aus.
\begin{itemize}[label=$\blacktriangleright$]
	\item Beschreiben Sie mit Hilfe des Graphen, warum eine Glühlampe als Beleuchtungsmedium nicht effizient ist.
	\item Bestimmen Sie mit Hilfe des Wien'schen Verschiebungsgesetzes die Lage des Maximums des angezeigten Schwarzkörperspektrums.
\end{itemize}
\subsection{Erzeugung verschiedener Farbtemperaturen mit einem Glühlämpchen}
Nehmen Sie drei verschiedene Spektra durch Änderung der Spannung (max. 10 Volt!) zu den subjektiv empfundenen Intensitäten "schwach", "mittel" und "stark" auf.

Ermitteln Sie in QtiPlot die Farbtemperaturen und speichern Sie die Diagramme ab.
\begin{itemize}[label=$\blacktriangleright$]
	\item Begründen Sie qualitativ, wie und warum sich die angelegte elektrische Spannung auf das aufgenommene Spektrum auswirkt.
\end{itemize}
\subsection{Auswertung des Sonnenspektrums}
\subsubsection{Aufnahme des Sonnenspektrums}
Richten Sie den Lichtwellenleiter so aus, dass Sie das Sonnenspektrum aufnehmen können (Aufnahme des Streulichts genügt). Speichern Sie bei aktuell gehaltenem Dunkelspektrum die Intensitätsverteilung ab.
\subsubsection{Bestimmung der Oberflächentemperatur der Sonne}
Rechnen Sie die Messdaten in QtiPlot um und stellen Sie sie graphisch dar. Erstellen Sie eine Fit-Kurve* und drucken Sie das Diagramm mit Angabe der ermittelten Oberflächentemperatur aus.

*Für den Fall, dass keine geeignete Kurve ermittelt werden kann, stellen Sie den Verlauf eines Schwarzkörperspektrums für $T=\SI{6000}{\kelvin}$ dar.
\begin{itemize}[label=$\blacktriangleright$]
	\item Begründen Sie kurz einige bei dieser Auswertung vorliegende Fehlerquellen.
\end{itemize}
\subsubsection{Bestimmung der wichtigsten Fraunhofer'schen Linien}
Stellen Sie das Sonnenspektrum in einem neuen Diagramm dar. Drucken Sie den Graphen aus. Bestimmen Sie mit Hilfe des "Datenlesers" die Wellenlängen der "Einbrüche" im Spektrum, welche den jeweiligen Fraunhofer'schen Linien entsprechen (vgl. Tabelle).		
\begin{enumerate}[label=$\blacktriangleright$]
	\item Kennzeichnen Sie die Lage diser Linien im Diagramm.
	\item Notieren Sie die gemessenen Wellenlängenwerte und bestimmen Sie die Differenz zu den Tabellenwerten.
	\item Wie lässt sich diese Differenz erklären?
\end{enumerate}
\begin{table}
	\caption{Wichtige Fraunhofer'sche Linien} \label{tab: flines}
	\centering
	\begin{tabular}{>{\bfseries}lll}
		\toprule[2pt]	Symbol&$\lambda$[nm]&Element\\
		\midrule[2pt]	A&$760.8$&\ce{O}\\
		\midrule		B&$686.7$&\ce{O}\\
		\midrule		C&$656.3$&\ce{H}\\
		\midrule		D\textsubscript{1}&$589.6$&\ce{Na}\\
		\midrule		D\textsubscript{2}&$589.0$&\ce{Na}\\
		\midrule		E&$527.0$&\ce{Fe}\\
		\midrule		F&$486.1$&\ce{H}\\
		\midrule		G&$430.8$&\ce{Fe}\\
		\midrule		H&$396.8$&\ce{Ca}\\
		\midrule		K&$393.4$&\ce{Ca}\\
		\bottomrule[2pt]
	\end{tabular}
\end{table}
\subsection{Bestimmung chemischer Elemente}
Stellen Sie das Spektrum einer Glimmlampe im Scope-Modus dar. Dieses erhalten Sie durch Aufnahme der Schalterbeleuchtung einer Steckdosenleiste. Justieren Sie hierzu den Lichtwellenleiter senkrecht über dem Schalter in wenigen Zentimetern Abstand.
\begin{enumerate}[label=$\blacktriangleright$]
	\item Erläutern Sie qualitativ, ob und wie der transparente gefärbte Kunststoff des Schalters die Aufnahme des Spektrums beeinflusst.
	\item Warum ist für diesen Versuchsteil keine Umrechnung der Daten in QtiPlot notwendig?
\end{enumerate}
Werten Sie das abzuspeichernde Spektrum in QtiPlot aus. Bestimmen Sie mittels ausliegender Messdaten bekannter Stoffe das für die Strahlung verantwortliche Element in der Lampe.

Importieren Sie zusätzlich die Messdaten des passenden Vergleichsspektrums und stellen Sie dieses im gleichen Diagramm dar.
\begin{enumerate}[label=$\blacktriangleright$]
	\item Welchen Vorteil hat dieses Verfahren gegenüber der Auswertung mit Hilfe von Tabellenwerten?
\end{enumerate}
Beschriften Sie markante Peaks mit zugehörigen Wellenlängen und drucken Sie das Diagramm aus.

(\#)Belegen Sie graphisch, dass in einer Energiesparlampe Anteile von Quecksilber vorhanden sind. Drucken Sie anschließend Ihre Ergebnisse mit geeigneter Beschriftung aus.
\subsection{Additive Farbmischung}
Betrachten Sie im Scope-Modus das Spektrum der farbwechselnden LED-Lampe und beantworten Sie folgende Fragen:
\begin{enumerate}[label=$\blacktriangleright$]
	\item Wie viele verschiedenfarbige LEDs werden in dieser Lampe eingesetzt?
	\item Welche (Peak-)Wellenlängen und Farben haben diese LEDs?
	\item Nennen Sie zusätzlich ein Beispiel, an dem ersichtlich wird, warum aufgrund eines Farbeindrucks nicht auf die zugrundeliegenden Wellenlängen geschlossen werden kann.
\end{enumerate}
Veranschaulichen Sie sich die additive Farbmischung mittels LEDs, deren Helligkeit einzeln verändert werden kann. Versuchen Sie, Mischfarben zu erzeugen. Verwenden Sie die Lampe als Streumedium. Richten Sie die LEDs auf eine weiße Fläche und erzeugen Sie Schatten in verschiedenen Farben.
\begin{enumerate}[label=$\blacktriangleright$]
	\item Erklären Sie das Zustandekommen dieser bunten Schatten.
\end{enumerate}
\chapter{Lichtbeugung an Spalt und Gitter}
In diesem Versuch soll das auch in der Spektroskopie Anwendung findende Prinzip der Beugung elektromagnetischer Wellen an einem oder mehreren Spalten genauer untersucht werden. Hierzu wird mit monochromatischem Licht gearbeitet, sodass aus der Theorie bekannte Beugungsgesetze, die in der Regel von der Wellenlänge des benutzten Lichts abhängen, möglichst gut überprüft werden können. Hier sollen zunächst die Zusammenhänge für den Einzelspalt angegeben werden:
\begin{align*}
	\sin(\phi_{k,min})&=\frac{\lambda}{b}\cdot k\\
	\sin(\phi_{k,max})&=\frac{\lambda}{b}\cdot\Big(k+\frac{1}{2}\Big)\\
	\phi_{0,max}&=0
\end{align*}
Hierbei ist $\lambda$ die Wellenlänge, $b$ die Spaltbreite und $k\in\IN$ die Beugungsordnung.

Fügt man einen zweiten Spalt (mit gleicher Spaltbreite) im Abstand $d$ hinzu, so erhält man zusätzlich zu den Beugungsextrema Interferenzeffekte, die ebenfalls zu Extrema führen, die sich folgendermaßen berechnen:
\begin{align*}
	\sin(\Psi_{m,min})&=\frac{\lambda}{d}\Big(m+\frac{1}{2}\Big)\\
	\sin(\Psi_{m,max})&=\frac{\lambda}{d}\cdot m
\end{align*}
Hierbei ist $m\in\IN_{0}$

Erweitert man diesen Aufbau erneut zu einem Gitter mit $N$ gleich breiten Spalten im gleichen Abstand, so ergibt sich eine etwas unhandlichere Beziehung für die Intensität des Lichtes in Richtung $\phi$:
\begin{displaymath}
	I(\phi)=\Bigg(\frac{\sin\Big(\frac{\pi b}{\lambda}\sin\phi\Big)}{\frac{\pi b}{\lambda}\sin\phi}\Bigg)^2\Bigg(\frac{\sin\Big(\frac{N\pi b}{\lambda}\sin\phi\Big)}{\sin\Big(\frac{\pi b}{\lambda}\sin\phi\Big)}\Bigg)^2
\end{displaymath}
\section{Vorbereitung}
\begin{enumerate}
	\item Nehmen Sie an, die Beugung findet nicht in Luft $(n\approx1)$, sondern in Wasser statt $(n>1)$. Wie ändert sich das Beugungsbild?
		\subitem Da Wasser ein optisch dichteres Medium ist, als Luft, wird die Wellenlänge verringert:
		\begin{displaymath}
			\lambda'=\frac{c'}{\nu}=\frac{c}{\nu}\cdot\frac{1}{n}=\frac{\lambda_0}{n}
		\end{displaymath}
		Dies bedeutet, da $\phi\in[0,\frac{\pi}{2}]$, dass die Winkel, unter denen die Extrema auftreten, kleiner werden.
	\item Es werde zuerst das Beugungsbild eines Doppelspaltes fotografisch aufgenommen; auf einem gleichartigen Film werden dann nacheinander die Beugungsfiguren beider Einzelspalte auf demselben Film aufgenommen. Insgesamt werden beide Filme gleich lange belichtet. Vergleichen Sie die Beugungsbilder miteinander. Erklären Sie Gleichheit oder Ungleichheit.
		\subitem Die Beugungsbilder unterscheiden sich, da die physikalischen Vorgänge nicht vollkommen gleich sind. Werden die Bilder der Einzelspalte aufgenommen, so addieren sich schlichtweg die Intensitäten:
		\begin{displaymath}
			I=I_1+I_2
		\end{displaymath}
		Beim Doppelspalt werden hingegen die Felder addiert, nicht die Intensität. Diese ist proportional zum Feldquadrat:
		\begin{displaymath}
			I\propto\vec{E}^2=(\vec{E}_1+\vec{E}_2)^2=\underbrace{\vec{E}_1^2}_{\propto I_1}+\underbrace{\vec{E}_2^2}_{\propto I_2}+2\vec{E}_1\cdot\vec{E}_2
		\end{displaymath}
		Die Gesamtintensität ist beim Doppelspalt daher nicht identisch mit der zweier Einzelspalten.
	\item Nehmen Sie an, bei einem Doppelspalt werden die beiden Spalte jeweils von verschiedenen Lasern beleuchtet. Wie würde sich das Beugungsbild gegenüber dem üblichen Experiment ändern?
		\subitem Wenn die Laser die gleiche Wellenlänge abstrahlen, ergibt sich das (bis auf durch eventuelle Phasenunterschiede hervorgerufene Schwebungen) gleiche Beugungsbild, wie bei der normalen Beugung des Lichtes einer Lichtquelle an einem Doppelspalt. Weisen die Laser unterschiedlicher Wellenlängen auf, so wirkt jeder Spalt als Einzelspalt für die jeweilige Lichtquelle.
	\item Nehmen Sie an, ein Laserstrahl wird durch Spiegel aufgespalten und die beiden Strahlen beleuchten je einen Spalt. Besteht ein Unterschied zu dem vorher geschilderten Fall? Wenn ja, erklären Sie, weshalb.
		\subitem Wird der Laser aufgespalten, so können Phasenunterschiede zwischen den Teilstrahlen entstehen, was wie im obigen Fall zu Schwebungen führen kann. Ansonsten ergibt sich das gleiche Bild
	\item Wie ändert sich das Beugungsbild eines Spaltes, wenn dieser statt mit einem Laser mit Licht einer Hg-Dampflampe beleuchtet wird?
		\subitem Das Licht einer Quecksilberlampe ist nicht monochromatisch, sondern weißt im sichtbaren Bereich gleich 6 unterschiedliche Emissionslinien auf, die daher auch unterschiedlich gegbeugt werden. Daher kann man im Beugungsbild Extrema unterschiedlicher Wellenlängen an unterschiedlichen Orten beobachten.
	\item Was unterscheidet Fraunhofer- und Fresnel-Beugung?
		\subitem Die Fraunhofer-Näherung der Lichtbeugung ist eine Fernfeldnäherung, die Fresnel-Näherung hingegen eine Nahfeldnäherung. Dies hat zur Folge, dass beispielsweiße das Beugungsintegral unter der Fraunhofer'schen Betrachtung recht einfach zu lösen ist, da es lediglich die Form einer Fourier-Transformierten hat. In Fresnel-Näherung hingegen ist dies nicht der Fall, und das Beugungsintegral ist im Allgemeinen nur numerisch zu lösen.
	\item Leiten Sie für den Einfachspalt die Formel $I(\phi)=I_0\Big(\frac{\sin\frac{\theta}{2}}{\frac{\theta}{2}}\Big)^2$ mit $\theta=\frac{2\pi}{\lambda}\cdot b\cdot\sin\phi$ und $I_0=I(\phi=0)$ für die Intensitätsverteilung in Abhängigkeit vom Beugungswinkel $\phi$ ab.
	
	Berechnen Sie das Intensitätsverhältnis $I(\phi_{k,max})/I(\phi=0)$ für die erste $(k=1)$ und zweite $(k=2)$ Beugungsordnung.
		\subitem Das elektrische Feld am Schirm erhält man zunächst durch Fouriertransformation:
		\begin{multline*}
			E(k_x,k_y)=E_0\int_{-\infty}^{\infty}\d x\int_{-\infty}^{\infty}\d y\Sigma_{Spalt}\e{-\imath k_xx}\e{-\imath k_yy}=E_0\delta(k_y)\cdot\int_{-\frac{b}{2}}^{\frac{b}{2}}\e{-\imath k_xx}\d x\\
			=E_0\delta(k_y)\frac{\e{-\imath k_x\frac{b}{2}}-\e{\imath k_x\frac{b}{2}}}{k_x}=E_0\cdot b\cdot\delta(k_y)\frac{\sin(k_x\frac{b}{2})}{k_x\frac{b}{2}}\\
			\Rightarrow I(k_x)=E(k_x)^2=I_0\frac{\sin^2(k_x\frac{b}{2})}{(k_x\frac{b}{2})^2}
		\end{multline*}
		Mit $k_x=\sin\phi\cdot\frac{2\pi}{\lambda}$ und $\theta=\frac{2\pi}{\lambda}\cdot b\cdot\sin\phi$ folgt:
		\begin{displaymath}
			I(\phi)=I_0\frac{\sin^2(\frac{\pi\cdot b}{\lambda}\sin\phi)}{(\frac{\pi\cdot b}{\lambda}\sin\phi)^2}=I_0\frac{\sin^2(\frac{\theta}{2})}{(\frac{\theta}{2})^2}
		\end{displaymath}
		Für die $k$-te Beugungsordnung gilt:
		\begin{displaymath}
			\sin(\phi_{k,max})=\frac{\lambda}{b}\Big(k+\frac{1}{2}\Big)
		\end{displaymath}
		Für die zu berechnenden Intensitätsverhältnisse erhält man somit:
		\begin{align*}
			\frac{I(\phi_{k,max})}{I_0}&=\frac{I_0\frac{\sin^2(\frac{\pi b}{\lambda}\sin\phi)}{\frac{\pi b}{\lambda}\sin\phi}}{I_0}=\frac{\sin^2(\frac{\pi b}{\lambda}\cdot\frac{\lambda}{b}(k+\frac{1}{2}))}{(\frac{\pi b}{\lambda}\cdot\frac{\lambda}{b}(k+\frac{1}{2}))^2}=\frac{\sin^2(\pi(k+\frac{1}{2}))}{(\pi(k+\frac{1}{2}))^2}\\
			\frac{I(\phi_{1,max})}{I_0}&=\frac{\sin^{2}(\pi(1+\frac{1}{2}))}{(\pi(1+\frac{1}{2}))^2}=\frac{4}{9\pi^2}\approx 0.045\\
			\frac{I(\phi_{2,max})}{I_0}&=\frac{sin^2(\pi(2+\frac{1}{2}))}{(\pi(2+\frac{1}{2}))^2}=\frac{4}{25\pi^2}\approx 0.016
		\end{align*}
	\item Verifizieren Sie für den Doppelspalt den Ausdruck $I(\phi)=4\cdot I_0\cdot(\frac{\sin\frac{\theta}{2}}{\frac{\theta}{2}})^2\cdot\cos^2\frac{\delta}{2}$ mit $\theta=\frac{2\pi}{\lambda}\cdot b\cdot\sin\phi$ und $\delta=\frac{2\pi}{\lambda}\cdot d\sin\phi$ und $I_0$ aus Frage 7. Begründen Sie anschaulich das Auftreten des Faktors 4 und berechnen Sie die Intensität des ersten Nebenmaximums $m=1$ relativ zum nullten in Abhängigkeit von Spaltbreite $b$ und Spaltabstand $d$. Für welches Verhältnis $d/b$ fällt das fünfte Nebenmaximum mit dem ersten Haupt-Minimum zusammen?
		\subitem Wie bereits oben beschrieben, werden beim Doppelspalt die Felder der hier um $\pm\frac{d}{2}$ verschobenen Einzelspalte überlagert. Bei der Fouriertransformation taucht nun ein zusätzlicher Phasenterm auf:
		\begin{displaymath}
			E_{Spalt,\nu}=E_{Spalt}\cdot\e{-\imath k_xx\frac{d}{2}}
		\end{displaymath}
		Für den Doppelspalt ergibt sich somit:
		\begin{displaymath}
			E_{DS}=E_{Spalt}\Big(\e{-\imath k_xx\frac{d}{2}}+\e{\imath k_xx\frac{d}{2}}\Big)=2\cdot E_0\frac{\sin(\frac{\theta}{2})}{\frac{\theta}{2}}\cdot\cos\Big(\frac{\pi\cdot d}{\lambda}\sin\phi\Big)
		\end{displaymath}
		Mit dem angegebenen $\delta$ berechnet sich die Intensität durch Quadrieren der Feldstärke zu
		\begin{displaymath}
			I_{DS}=4\cdot I_0\cdot\Big(\frac{\sin\frac{\theta}{2}}{\frac{\theta}{2}}\Big)^2\cdot\cos^2\Big(\frac{\delta}{2}\Big)
		\end{displaymath}
		Der Faktor $4$ taucht also auf, weil zunächst zwei Felder gleichre Amplitude addiert werden, und das resultierende Feld $2E_0$ anschließend quadriert wird.
		
		Für das Nebenmaximum $m$-ter Ordnung gilt:
		\begin{displaymath}
			\sin\Psi=\frac{\lambda}{d}\cdot m
		\end{displaymath}
		Damit erhält man für das Verhältnis der Intensitäten des ersten zum nullten Nebenmaximum analog zu Aufgabe 7:
		\begin{displaymath}
			\frac{I(\Psi_{1,max})}{I_0}=\Bigg(\frac{\sin(\frac{\pi b}{d})}{\frac{\pi b}{d}}\Bigg)^2
		\end{displaymath}
		Damit das 5. Nebenmaximum mit dem erten Hauptminimum zusammenfällt, muss gelten:
		\begin{displaymath}
			\frac{\lambda}{d}\cdot 5=\frac{\lambda}{b}\Rightarrow\frac{d}{b}=5
		\end{displaymath}
\end{enumerate}
\section{Durchführung}
\subsection{Beugungsbild des Einfachspaltes}
\begin{enumerate}[label=\alph*)]
	\item Nehmen Sie die Intensitätskurve der Beugungsfigur eines Einfachspaltes auf.
	\item Berechnen Sie aus Ihren Messdaten die Spaltbreite. Finden Sie eine andere optische Messmethode zur Bestimmung der Spaltbreite und vergleichen Sie beide Ergebnisse miteinander.
	\item Werten Sie auch die Intensitätsverhältnisse aus und vergleichen Sie die Ergebnisse mit der Beugungstheorie.
\end{enumerate}
\subsection{Beugungsbild des Doppelspaltes}
\begin{enumerate}[label=\alph*)]
	\item Wiederholen Sie die obige Messung für einen Doppelspalt. Was fällt am Beugungsbild, was an der Intensitätskurve des Beugungsbildes auf?
	\item Erklären Sie die Intensitätskurven mit der Beugungstheorie.
	\item Kontrollieren Sie das Verhältnis der Intensitäten von 0. und 1. Maximum 2. Klasse. Vergleichen Sie mit der Theorie. Überlegen Sie alle Fehlermöglichkeiten, um Abweichungen von Experiment und Theorie zu erklären.
\end{enumerate}
\subsection{Beugungsbild eines optischen Gitters}
\begin{enumerate}[label=\alph*)]
	\item Nehmen Sie die Intensitätskurve der Beugungsfigur eines optischen Gitters auf.
	\item Vergleichen Sie die Kurve für den N-fachen Spalt mit der Kurve des Doppelspalts.
	\item Berechnen Sie aus den Messdaten den mittleren Spaltabstand (Gitterkonstante)
	\item Versuchen Sie, den Einfluss des Einzelspaltes auf das Beugungsbild zu sehen.
\end{enumerate}
\chapter{Optische Geräte}
Bei diesem Versuch soll selbständig mit einfachen, aber grundsätzlich wichtigen optischen Geräten experimentiert werden. Im Vordergrund steht hier also das eigene Beobachten. Die quantitative Auswertung beruht auf den Gesetzten der geometrischen Optik. Konkret lässt sich dies an folgenden Aufgaben festmachen:
\begin{itemize}
 \item  Theoretische Überlegungen und Berechnungen experimentell überprüfen
 \item Grundgesetze der geometrischen Optik durch Anwenden besser verstehen
 \item  Probleme der geometrischen Optik praktisch lösen
 \item  Versuchsaufbauten justieren und optimieren
 \item Wichtige optische Geräte und Instrumente (Lupe, Fernrohre, Projektionsapparat, Mikroskop)
 in offener Versuchsanordnung nachbauen
\end{itemize}
Es wird mit folgenden Geräten experimentiert:
\begin{itemize}
 \item Lupe: Konvexlinse, die ein vergrößertes virtuelles Bild des Gegenstandes erzeugt, der sich innerhalb der Brennweite befindet
 \item Astronomisches Fernrohr: (auch Keplersches Fernrhr genannt) besteht aus zwei Sammellinsen
 \item Terrestrisches Fernrohr: besteht aus drei Linsen, wobei die Zwischenlinse der Bildumkehr dient
 \item Das holländische Fernrohr (Galileiisches Fernroht): aufgebaut aus Sammel- und Zerstreuungslinse
 \item Diaprojektor: aufgebaut aus Lampe, Kondensator, Objektiv und Schirm
 \item Erzeugung eines vergrößerten, auf dem Kopf stehenden, seitengedrehten Bild
 \item zwei Strahlengänge: beleuchtend und abbildend
 \item Mikroskop: Entwerfung eines reelen Zwischenbildes in vergößertem Maßstab, welches mit einer Lupe beobachtet wird
\end{itemize}
\newpage
\section{Fragen zur Vorbereitung}
1. Wie lautet die Abbildungsgleichung für dünne Sammel- und Zerstreuungslinsen? Welche Näherungen werden bei der Herleitung gemacht? Was ist die Hauptebene einer Linse?\\
Abbildungsgleichung für dünne Linsen:   $\frac{1}{b}+\frac{1}{g}=\frac{1}{f}=(n-1)\left(\frac{1}{r_1-r_2}\right)$ falls $d \ll r_1, r_2$\\
g= Gegenstandsweite \\
b= Bildweite\\
f= Brennweite\\ \\
Hauptebene: zwei in einem Abbildungssystem definierte Ebenen, in denen vereinfacht die Brechungen der Lichtstraheln angenommen werdenm im Raum zwischen den Hauptebenen werden die Lichtstrahlen parallel zur optischen Achse verlaufend gedacht\\
2. Wodurch werden die Bildhelligkeit und das Gesichtsfeld beeinflusst bzw. begrenzt?\\
Gesichtsfeld = allle zentralen peripheren Gegenstände und Punkte des Außenraums, die bei ruhiger, gerader Kopfhaltung und geradeaus gerichteten, bewegungslosem Blick wahrgenommen werden können, auch ohne sie direkt zu fixieren\\
Umfang des Gesichtsfeldes ist abhängig von der Pupillenweite und der Lage des Auges zu den Nachbarorganen (Nase, Augenlid). Bei teif in der Augenhöhle sitzenden Augen ist das Gesichtsfeld kleiner. An der Innenseite ist die größte Ausdehnung, an der Innenseite die kleinste. Die Farbenempfindlichkeit ist nicht an allen Stellen gleich - die äußerste Peripherie des Gesichtsfeldes ist die "farbenlose Zone", nach der Mitte zu folgt des Gesichtsfeld von blau, gelb, rot und grün. \\
Bildhelligkeit = Beleuchtungsstärke in der Bildebene eines abbildenden Systems oder Grauwert (digitales Bild)
Bild wird heller, je größer der Linsendurchmesser ist, d.h. je mehr von der Strahlungsenergie einfangen werden kann. \\\\
3. Leiten Sie die Formeln für die Vergrößerung von Lupe, den Fernrohren und dem Mikroskop
her. Berücksichtigen Sie beim Fernrohr insbesondere die großen Gegenstandsweiten.\\
Allgemein gilt:
\begin{align*}
%\begin{center}
V=\frac{\text{Sehwinkel mit Instrument}}{\text{Sehwinkel ohne Instrument}}=
\frac{\text{Sehwinkel mit Instrument}}{\text{Sehwinkel im Abstand $S_0$}}=
\frac{\epsilon}{\epsilon_0} \\
V_{Lupe}=\frac{\epsilon}{\epsilon_0}=\frac{G}{f_l}*\frac{S_0}{G}=\frac{S_0}{f_l} \\
V_{Mikroskop}=V_{Objektiv} \cdot V_{Okular}=\frac{l \cdot S_0}{f_{Obj} \cdot f_{Oku}}\\ (l=Tubulänge)\\
V_{Fernrohr}= V_{Objektiv} \cdot V_{Okular}= \frac{\frac{B}{S_0}}{\frac{G}{g}}=\frac{f_{Objekiv}}{f_{Okular}}\\
V_{Holländisches Fernrohr}=\frac{f_{Objektiv}}{|f_{Okular}|}\\
%\end{center}
\end{align*}
\\4. In welche Entfernungsbereiche - bezogen auf die Brennweite der Lupe - können der zu betrachtende Gegenstand und das Auge gebracht werden?\\
Bei einer Lupe beträgt der ideale Abstand zum Auge 25 cm, wenn der Gegenstand im Brennpunkt liegt. Liegt der Gegenstand zwischen Brennpunkt und Linse, ist das Auge angespannt, man muss das Auge auch näher an die Linse bringen. Falls sich der Gegenstand hinter der Brennweite befindet, kann er nicht mehr vergrößert gesehen werden.\\
\\5. Warum benutzt man in der Praxis meist Prismenfernrohre? Zeichnen Sie den Strahlengang!\\
\begin{figure}[h]
 \centering
 \subfigure[Strahlengang bei Umkehrprismen] {\includegraphics[width=0.49\textwidth]{prismenumkehr1}}
 % \caption{Gesichtsfeldblende und Umkehrprismen}
 \subfigure[Bildumkehr]{\includegraphics[width=0.49\textwidth]{prismenumkehr2}}
\end{figure}
Das Prismenfernrohr besteht wie das astronomische Fernrohr aus Obetiv und Okular. Das astronomische Fernrohr ergibt aber ein kopfstehtendes Bild. Zur Umkehrung dieses Bildes werden Prismen benutzt. Da bei dem kopfstehenden Bild auch links und rechts vertauscht werden, muss der Strahlengang noch ein zweites Prisma durchlaufen, damit auch die Seitenrichtigkeit wiederhergestellt ist. Diese Prismenanordnung bringt gegenüber dem terrestrischen Fernrohr durch den 3fach nebeneinanderglegten Strahlengang eine wesentlich verkürzte Baulänge. Der gegenüber dem Augenabstand vergrößerte Objektivabstand ist für das räumliche Sehen von Vorteil. \\ \\
6. Erläutern Sie anhand einiger einfacher optischer Abbildungsanordnungen die Begriffe Apertur und Gesichtsfeldblende. Was will man mit ihnen bezwecken?\\
\underline{Gesichtsfeldblende:} gibt der optischen Abbildung eine scharfe Begrenzung. Der abgebildete Bildausschnitt kann somit verändert werden, die Helligkeit des Bildes bleibt unbeeinflusst (im Gegensatz zur Aperturblende)  \\
\underline{Aperturblende}: begrenzt bei einem optischen System dessen Apertur (Öffnungsweite), bei Verkleinerung der Apertur werden Helligkiet und AUflösung geringer, die Schärfentiefe wird größer und der Bildausschnit bleibt erhalten (beim Auge ist die Iris die Aperturblende)
\begin{figure}[h]
\centering
 \subfigure[Gesichtsfeldblende und Umkehrprismen] {\includegraphics[width=0.49\textwidth]{gesichtsfledblende}}
% \caption{Gesichtsfeldblende und Umkehrprismen}
 \subfigure[Aperturblende]{\includegraphics[width=0.49\textwidth]{Apertur}}
\end{figure}

7. Wie lassen sich in optischen Systemen mit einem vorgegebenen Linsensatz kurze Brennweiten
erzielen?\\
Es gilt: $\frac{1}{f}=\frac{1}{f_1}+\frac{1}{f_2}-\frac{d}{f_1f_2}$\\
$\rightarrow$ Kurze Brennweiten lassen sich durch einen möglichst kleinen Abstand d erzielen\\
\\8. Welche Linsenfehler gibt es? Nennen Sie Möglichkeiten zu ihrer Beseitigung!\\
\underline{sphärische Aberration:} bewirkt Unschärfe des Bildes; Beseitigung durch Unterdrückung achsenferner Strahlen mithilfe einer Blende\\
\underline{Astigmatismus:} Objekte, die außerhalb der optischen Achse liegen, werden unscharf abgebildet; Korrektur durch Kombination von sphärischen und torischen Gläsern. Die Hornhautverkrümmung kann auch operativ behandelt werden.\\
\\9. Erläutern Sie, warum Präzisionsfernrohre für die Astronomie nicht mit Linsen sondern mit
Spiegeln gebaut werden.\\
Es werden Spiegel benutzt, da sie eine größere Fläche abbilden können und billiger sind. Außerdem kann es keinen Farbfehler geben.\\\\
10. Zeichnen und erläutern Sie die Optik im menschlichen Auge! Erörtern Sie Kurz- und Weitsichtigkeit sowie Astigmatismus.\\
\begin{figure}[h]
 \includegraphics[width=0.5\textwidth]{Auge}
 \caption{Anatomie des Auges}
\end{figure}
Hornhaut: schützt das Auge nach außen\\
Iris und Pupille: Iris regelt Größe der Pupille und damit die Menge des durch die Linse durchtretenden Lichts\\
Linse: wesentliches Abbildungssystem des Auges\\
Glaskörper: Bestandteil des Abbildungssystems, sorgt für konst. Abstand Linse + Netzhaut\\
Netzhaut: beinhaltet Sehsinneszellen (Zäpfchen + Stäbchen)\\
Aderhaut: enthält Versorgungssystem für Netzhaut\\
Lederhaut: schützt das Auge nach außen
\underline{Kurzsichtigkeit:} zu langer Augapfel, der Fokus liegt vor der Netzhaut\\
\underline{Weitsichtigkeit:} zu kurzer Augapfel, der Fokus liegt hinter der Netzhaut\\
\underline{Astigmatismus:} siehe Aufgabe 8, Ursache: Hornhaut nicht exakt kreisförmig, sondern torisch, unterschiedliche Brechkraft an unterschiedlichen Stellen der Hornhaut, Strahlen bündeln sich nicht an einem Punkt
\\\\
11. Warum besteht der Kondensor eines Diaprojektors aus zwei plankonvexen Linsen? (Hinweis:
Berücksichtigen Sie die Reflexionsverluste!)
Plankonvexe Linsen werden benutzt, um die Dias besser auszuleuchten, ein möglichst großer Teil des Lichts der Projektionslampe soll in den abbildenden Strahlengang eingebracht werden, Verringerung der sphärischen Aberration und Totalreflexion\\
\\12. Wie kann man die Vergrößerung eines Fernrohrs ohne Kenntnis der Brennweiten messen?\\
Die Vergrößerung eines Fernrohres kann man mithilfe der Durchmesser ein- und ausfallenden Strahlenbündel bestimmen.\\
$V=\frac{d_{Aus}}{d_{Ein}}$ \\
\\13. Schätzen Sie die Größenordnung der Objektivbrennweite für einen Heimdiaprojektor ab, wenn Sie Raumgröße, Dia- und Leinwandgröße in Betracht ziehen\\
Leinwandgröße: 5m, Diagröße 5cm, Abstand zwischen linse und Dia: 10cm, Abstand zwischen LInse und Wand: 10m
Für Lupe gilt: $V=\frac{B}{G} \approx\frac{5m}{5cm}=100$ \\
Für Brennweite gilt: $f=(\frac{1}{0.1m}+\frac{1}{10m})^{-1}=9.9cm$
\\
\section{Durchführung}
\subsection{Lupe}
Betrachten Sie Gegenstände mit und ohne Lupe. Bestimmen Sie bei den verschiedenen Beobach
tungsarten jeweils die subjektiv ermittelten Vergrößerungen an.
\subsection{Astronomisches Fernrohr}
Material: 2 Achromate f = 500 mm, 60 mm und f = 40 mm, 10 mm Bikonvexlinse f = 500 mm, 38
mm 2 Bikonvexlinsen f = 40 mm, = 24 mm 2 Iris-Blenden, Ständer mit Schiene und Reiter
\begin{enumerate}
 \item   Bauen Sie ein astronomisches Fernrohr von mehr als 10-facher Vergrösserung aus zwei Bikon
 vexlinsen.
 \item Benutzen Sie eine Gesichtsfeldblende. Wo muss sie stehen, damit eine Gesichtsfeldbegrenzung
 bei Betrachtung sehr weit entfernter Gegenstände entsteht?
 \item  Benutzen Sie als Objektivlinse eine achromatische Linse. Beschreiben Sie die Unterschiede in
 der Bildgüte und bestimmen Sie die Vergrösserung experimentell.
 \item Stellen Sie eine Blende vor die Objektivlinse und betrachten Sie das Bild mit verschiedenen
 Durchmessern der Blende vor dem Objektiv
 \item Benutzen Sie als Okularlinse einen Achromaten
\end{enumerate}
\subsection{Terrestrisches Fernrohr}
Bauen Sie ein terrestrisches Fernrohr gem. Abb. og.3 auf. Diskutieren Sie insbesondere die Rolle
der mittleren Linse und ihren Einfluss auf die Vergrösserung. Untersuchen Sie die Eiflusse von
Linsenqualität (Achromate) und Blenden wie beim Astronomischen Fernrohr
\subsection{Holländisches Fernrohr}
Material: Achromat f = 500 mm, Bikonkavlinse f = -100 mm Bikonkavlinse f = -200 mm, Blende,
Ständer mit Schiene und Reitern
\begin{enumerate}
 \item . Welchen Eifluss hat der Objektivdurchmesser auf den Gesichtsfelddurchmesser?
 \item Bestimmen Sie den Durchmesser des Gesichtsfeldes bei 5-facher Vergrösserung und bei 2,5
 facher Vergrösserung (in beiden Fällen gleicher Objektivdurchmesser von 32 mm). Welcher Zu
 sammenhang besteht zwischen dem Durchmesser des Gesichtsfeldes und der Vergrösserung?
 \item Bestimmen Sie in beiden Fällen die Vergrösserung experimentell.
 
\end{enumerate}
\subsection{Spiegelteleskop}
Beobachten Sie weit entfernte Gegenstände mit dem Spiegelteleskop. Beurteilen Sie Vor- und Nach
teile gegenüber den vorher gebauten Fernrohren.
\subsection{Diaprojektor}
Vorhandene Komponenten: Bikonvexlinse f = 80 mm Optische Bank Achromat f = 80 mm Lampe
Kondensor f = 130 mm Diahalter mit Diapositiv Iris-Blende Schirm
Erzeugen Sie ein vergrößertes Bild des Diapositivs auf dem Schirm.
\begin{enumerate}
 \item  Beginnen Sie ohne Kondensorlinse mit einer Bikonvexlinse als Objektiv. Bilden Sie zunächst
 möglichst scharf ab bei einer Bildgrösse von ca. 10 cm. Verschieben Sie jetzt den Schirm so,
 dass das Bild etwas unscharf wird und verkleinern Sie dann mit der Irisblende den Linsen
 durchmesser. Beschreiben Sie die beobachteten Effekte.
 \item Benutzen Sie nun bei voller Objektivöffnung den Kondensor zur Beleuchtung des Diapositives
 und beschreiben Sie die Veränderungen im Bild gegenüber Aufgabe 1.
 \item Ändern Sie wieder wie in Aufgabe 1 den Objektivdurchmesser. Beschreiben Sie die beobach
 teten Effekte.
 \item Benutzen Sie nun als Objektiv eine korrigierte Linse (Achromat) mit gleicher Brennweite unter
 Konstanthalten der Abstände. Diskutieren Sie die Bildfehler (Farbfehler, Verzerrungen, fehlen
 de Schärfentiefe).
 
\end{enumerate}
\subsection{Mikroskop}
Geräte: Lampe (mit Kondensor), Dia mit Strichgitter d = 0,1 mm halbdurchlässiger Spiegel mit Halterung, Beobachtungsschirm Okular f = 25 mm (V = 10 ), Objektiv f = 40 mm, 10 mm Berechnen Sie die optischen Parameter eines Mikroskops für die Vergrösserung V = 50. Bestimmen Sie die Vergrösserung experimentell mit Hilfe der modifizierten Mikroskopanordnung in Abb.10: Nach der Okularlinse wird eine Sammellinse L (fL = 100 mm) platziert. Ein Beobachtungsschirm S wird im Abstand der Brennweite fL aufgestellt. Durch die Wirkung dieser Sammellinse werden die parallelen Lichtbündel, die die Okularlinse unter
dem Winkel $\psi$ verlassen auf einen Punkt am Beobachtungsschirm fokussiert. Die Vergrösserung V
des Mikroskopes kann bestimmt werden aus dem Verhältnis des Winkels $\psi$
und dem Winkel, unter
dem der Gegenstand G dem ünbewaffneten Augeïm Abstand der deutlichen Sehweite, s erscheint:
\begin{center}
 $V=\frac{\psi}{arctan g/s}$
\end{center}
Als Objekt benutzen Sie am besten das Strichgitter. Um den Winkelabstand zweier Gitterlinien auf
dem Schirm zu messen, bestimmen Sie die Höhe b, die von mehreren Gitterstrichen auf dem Beob
achtungsschirm eingenommen wird, und dividieren sie durch die Zahl der Zwischenräume zwischen
den betrachteten Gitterstrichen.

\chapter{Bestimmung des Planck'schen Wirkungsquantums}
In diesem Versuch soll das Planck'sche Wirkungsquantum $\planck$ bestimmt werden. Diese Größe ist neben der Lichtgeschwindigkeit $\sol$ und der Gravitationskonstanten $\G$ die dritte fundamentale Naturkonstante der Physik. Sie beschreibt für schwingfähige Systeme das konstante Verhältnis aus der kleinstmöglichen Energieänderung und der Schwingungsfrequenz. Daraus folgt insbesondere, dass solche Systeme nur ganzzahlige Vielfache des sog. Schwingungsquants $\Delta E=\planck\nu$ aufnehmen können. Auch der Drehimpuls eines Systems kann sich nur um ganzzahlige Vielfache von $\hbar\equiv\frac{\planck}{2\pi}$ ändern. Dies sind jedoch nur einige der wichtigen Zusammenhänge, in denen die Planckkonstante eine wichtige Rolle spielt.

Die Bestimmung dieser fundamentalen Konstanten soll in diesem Versuch durch Messungen an Leuchtdioden, insbesondere der Feststellung von deren Wellenlängen über die Gleichung 
\begin{equation}
	E=\planck\nu
\end{equation}
bestimmt werden. Zwischen der Wellenlänge $\lambda$ und der Frequenz $\nu$ besteht dabei der bekannte Zusammenhang: $\sol=\lambda\nu$.

Da solche LEDs Halbleiter sind, muss in diesem Versuch auch eine gewisse Vertrautheit mit Grundlagen der Halbleiterphysik gegeben sein. Für diesen Versuch ist insbesondere das Resultat wichtig, dass die Wellenlänge des abgestrahlten Lichts nur von der sogenannten Gap-Energie $E_{\mathtt{gap}}$ abhängt, also von der Energie, die frei wird, wenn ein Elektron vom Leitungsband, der Energiezone, in der sich frei bewegliche und damit leitungsfähige Ladungsträger befinden, in das energetisch tieferliegende Valenzband übergeht. Diese Energie ist eine Materialkonstante, womit klar wird, dass die Farbe der LED nur von den verwendeten Halbleitern abhängt:
\begin{equation}
	\lambda=\frac{\planck\cdot\sol}{E_{\mathrm{gap}}}
\end{equation}
\section{Vorbereitung}

\chapter{Gekoppelte Pendel}
Gekoppelte Pendel sind solche Pendel, zwischen denen ein Energieaustausch stattfindet. Die dadurch ausgeführten Schwingungen werden auch Koppelschwingung genannt. In jedem Pendel wirkt ein Richtmoment, das durch die Schwerkraft hervorgerufen wird und bestrebt ist, das Pendel in die Ruhelage zurückzuziehen. Außerdem macht sich die vorhandene Kopplung in Form eines zusätzlichen Richtmoments bemerkbar, das so wirkt, dass die Feder möglichst entspannt wird.
Häufig versteht man unter gekoppelten Pendeln zunächst die Wechselwirkung zwischen zwei Pendeln. Das Konzept lässt sich jedoch auf beliebig viele Pendel anwenden. Mehrere gleiche Pendel, die in einer Reihe angeordnet mit ihren unmittelbaren Nachbarn wechselwirken, bezeichnet man als Schwingerkette.
\\
Das Richtmoment D ist bei einer mechanischen Torsion die Proportionalitätskonstante zwischen dem anliegenden Drehmoment
M und dem Drehwinkel $\phi$ :
\begin{center}
$\vec{M}= D\vec{\phi}$\\
\end{center}
Die Kreisfrequenzen der harmonischen Pendelbewegungen:
\begin{center}
 $\omega_{gl} = \sqrt{\dfrac{g}{L}}$ \\
 $\omega_{geg} =\sqrt{\dfrac{g+2k}{L}}$
\end{center}
Schwebungsdauer $T_{s}$ :
\begin{center}
 $T_{s}= 2\cdot\dfrac{T_{gl}\cdot T_{geg}}
{T_{gl}-T_{geg}}$
\end{center}
Der Kopplungsgrad K ist ein Maß für die Stärke der Kopplung und ist gegeben durch:
\begin{center}
 $K= \dfrac{Dr^2}
 {D+Dr^2}$
\end{center}
\begin{minipage}{\textwidth}
 \centering
 \includegraphics[width=0.7\textwidth]{gekpendel}
 \captionof{figure}{Gekoppelte Pendel}
 \label{fig:Abb4}
\end{minipage}
\section{Fragen zur Vorbereitung}
1. Ist die Eigenkreisfrequenz
$\omega_{geg}$ bei gegenphasiger Schwingung kleiner, gleich oder größer als bei gleichsinniger Schwingung  $\omega_{gl}$?\\
Bei gleichsinniger Schwingung ändert sich die Länge der Feder und somit ihre Spannung nicht. Somit werden keine zusätzlichen Drehmomente auf die Pendel ausgeübt. Sie schwingen dann unabhängig voneinander: $ \omega_{gl}=\omega_{0}$ (Diese Frequenz würde sich auch ohne Kopplungsfeder einstellen.) \\
Bei gegensinniger Schwingung wird die Feder periodisch verformt und übt zusätzliche zeitlich veränderliche Drehmomente auf die Pendel aus. Je weiter z.B. beide Pendel nach außen schwingen, desto stärker werden sie von der Feder zurückgezogen, Es ergibt sich eine symmetrische Schwingung der beiden Pendel, deren Eigenkreisfrequenz  $\omega_{geg}$ durch die zusätzlichen Drehmomente allerdings größer ist als $\omega_{0}$\\
\\2. Welche Bedeutung haben gekoppelte Schwingungen in der Molekülphysik?\\
In Molekülen kommt es zu unterschiedlich starken Bindungen  zwischen den einzelnen beteiligten Atomen. Diese kann man sich als unterschiedlich starke Federn vorstellen. Also schwingt dann bei Anregung nicht nur ein Atom oder eine isolierte Bindung, sondern durch Kopplung der Bindungen auch die umliegenden Bindungen bzw. Atome.\\
\\3. Wie kann man Schwebungen zum Stimmen von Musikinstrumenten verwenden? Warum eignet sich diese Methode insbesonders bei tiefen Frequenzen?\\
Das Instrument wird solange auf den Kammerton als Referenzton eingestimmt, bis keine Schwebung mehr zu hören ist. Bei Schwebung treten Lautstärkeschwankungen auf. Bei hohen Frequenz sind sie zu schnell aufeinander folgend und somit schwerer zu erkennen. Deswegen eignet sich diese Methode besonders bei tiefen Frequenzen.\\
\\4. Wie kann man mit Schwebungen sehr hohe Frequenzen messen?\\
Voraussetzung ist, eine Schwingung gleicher Amplitude und ähnlicher Frequenz erzeugen zu können. Überlagert man die zu messende Schwingung mit der Referenzschwingung, so entsteht eine Schwebung, deren Schwebungsfrequenz sich weit unter der zu messenden Frequenz befindet und somit einfach zu bestimmen ist. Mit Hilfe der nun vorhandenen Schwebugsfrequenz und der Referenzfrequenz kann die noch unbekannte Frequenz erschlossen werden.\\
\\5. Wie kann mittels eines sog. FRAHMschen Schlingertanks die Schlingerbewegung eines großen Schiffs verringert werden (Schiffs-Stabilisator)?\\Ein Schlingertank ist ein mit Wasser gefüllter Tank im Rumpf eines Schiffes, der Schwankungen um die Längsachse (das sogenannte Rollen) dämpfen soll.\\
Der von Frahm im Jahr 1910 patentierte Flüssigkeitsdämpfer besteht aus einem Uförmigen Rohrsystem gefüllt mit  z.B. Wasser. Der Flüssigkeitsdämpfer diente ursprünglich zur Dämpfung der Rollbewegung von Schiffen und gilt als einer der ersten Schwingungsdämpfer. Durch die Strömung der Flüssigkeitssäule wird die gleiche dämpfende Wirkung wie bei den mechanischen Schwingungsdämpfern verursacht. Die Eigenfrequenz sowie das Dämpfungsverhalten des Flüssigkeitsdämpfers werden über die Geometrie des Rohrsystems bestimmt. \\
Im Wesentlichen ist es ein Wasserpendel, das auf die Eigenresonanz des Schiffes abgestimmt ist. Treffen seitlich Wellen auf das Schiff, sind im Resonanzfall die Phase der Schiffsschwingung und die der stoßenden Wellen um $\pi/2$ verschoben. Nun bewegt sich das Wasser im Tanksystem (Zwei miteinander verbundene Tanks an den Seiten des Schiffs).  Stimmt die Eigenfrequenz des Wasserpendels mit dem des Schiffs überein, ergibt sich noch eine weitere Phasenverschiebung um $\pi/2$ . Letztendlich addieren sich die beiden Phasenverschiebung zu $\pi=180 $ Grad, sodass die gegenphasige Bewegung die Schlingerbewegeung des Schiffes dämpft.
\section{Durchführung}
Bauen Sie ein gekoppeltes Pendel auf. Achten Sie insbesonders auf die
Nadellager der beiden Pendel: die Nadeln sind sehr spitz, daher empfindlich; darüber hinaus könnten
Sie sich bei unsachgemäßem Umgang an den Spitzen verletzen.
Die Kopplungsfeder wird vorerst noch nicht verwendet.
Verbinden Sie die Winkelsensoren mit dem Messverstärker (Verstärkungsfaktor 10). Schließen Sie
den Ausgang des Messverstärkers an das Oszilloskop und an das USB-Multimeter an. Über das USB
Multimeter können Sie die Daten mit einer Auflösung von 2 Hz in eine Datei schreiben. Die Daten
werden anschließend über die Datenvisualisierung qtiplot ausgewertet.
Achtung: der Dezimalpunkt des Datensatzes muss ggfs. in ein Dezimalkomma ersetzt werden,
damit es von qtiplot richtig interpretiert werden kann!
\begin{itemize}
 \item 1. Justieren Sie bei absolutem Stillstand der Pendel den Offset der Winkelsensoren auf 0,0V.
 \item 2. Lenken Sie nun eines der Pendel leicht aus und beobachten Sie die Schwingung. Nehmen Sie über die Winkelsensoren und den Messverstärker einen Datensatz auf. Bestimmen Sie die Schwingungsdauer $T_{01} =\dfrac{2 \pi }{\omega_{01}}$. Wiederholen Sie den Messvorgang für das zweite Pendel und stellen Sie sicher, dass im Rahmen der Messgenauigkeit $T_{01}= T_{02} $ gilt.
 \item3. Verbinden Sie nun die Pendel über die Kopplungsfeder miteinander. Notieren Sie sich die Position der Befestigungsstelle.
 \item4. Zeichnen Sie die Schwingung auf, wenn beide Pendel in Phase sind und bestimmen Sie daraus die Periodendauer $T_{gl}$.
 \item5. Zeichnen Sie die Schwingung auf, wenn beide Pendel gegenphasig sind und bestimmen Sie daraus die Periodendauer $T_{geg}.$
 \item6. Lenken Sie nun ein Pendel aus der Ruhelage aus und erzeugen so gekoppelte Schwingungen.
 Zeichen Sie die gekoppelten Schwingungen auf und bestimmen Sie daraus die Oszillationsperiode Tm und die Schwebungsdauer $T_S.$
 \item7. Vergleichen Sie die erhaltenen Werte mit denen, die Sie für die natürlichen Perioden $T_{gl}$ und $T_{geg}$ zuvor berechnet haben. Berücksichtigen Sie eine kurze Fehlerbetrachtung.
 \item 8. Bestimmen Sie den Kopplungsgrad, indem Sie ein Pendel um eine bestimmte Strecke aus der
 Ruhelage auslenken und die Auslenkung des anderen Pendels messen. Wiederholen Sie die
 Messung für mindestens drei unterschiedliche Auslenkungen, sofern Sie noch über ausreichend
 Zeit verfügen.
 \item 9. Wiederholen Sie Aufgabe 8 bis 6 für einen andern Kopplungsgrad, indem Sie die Feder an
 einem anderen Punkt der Pedelstange befestigen.
\end{itemize}
\chapter{Spezifische Ladung $e/m$ des Elektrons}
In diesem Teil des Praktikums soll die sogenannte spezifische Ladung des Elektrons bestimmt werden. Als spezifische Ladung bezeichnet man gemeinhin die Ladung pro Masse eines Teilchens. Diese Größe findet in mehreren Bereichen der Physik Anwendung, beispielsweise in der Massenspektrometrie, wo Analyten ionisiert und dann nach dem Kehrwert der spezifischen Ladung sortiert werden.

Hier ist allerdings die Möglichkeit wichtiger, über die Bestimmung des Verhältnisses $\frac{e}{m}$ auf die Masse des Elektrons zu schließen. Dies ist möglich, da zum Beispiel über den Millikan-Versuch die Elementarladung $e$ direkt bestimmt werden kann. 

Um mit den in diesem Versuch verfügbaren Aufbauten also die spezifische Ladung des Elektrons zu bestimmen, müssen zunächst einige mathematische Zusammenhänge für die involvierten physikalischen Größen bekannt sein:

Die Kräfte auf ein Elektron im elektromagnetischen Feld sind gegeben durch:
\begin{align}
	&\vec{F}=e\vec{E}&&\mathrm{Coulombkraft}\\
	&\vec{F}=-e\mu_r\mu_0\vec{v}\times\vec{H}=-e\vec{v}\times\vec{B}&&\mathrm{Lorentzkraft}
\end{align}
Nach Durchlaufen einer Potentialdifferenz U besitzt ein Elektron die potentielle Energie:
\begin{equation}
	W_{\mathrm{pot}}=eU
\end{equation}
Im Gegensatz dazu verrichtet die Lorentzkraft im Falle der Magnetostatik keine Arbeit am Teilchen. Aus der Energieerhaltung ergibt sich daher:
\begin{equation}
\label{eq:2} \frac{e}{m}=\frac{v^2}{2U}
\end{equation}
Da davon ausgegangen werden kann, dass die Bewegung des Elektrons in einer Ebene stattfindet, die senkrecht zu $\vec{B}$ ist, lässt sich weiterhin schreiben:
\begin{equation}
\label{eq:1} \frac{e}{m}=\frac{2U}{r^2B^2}
\end{equation}
\section{Vorbereitung}
\begin{enumerate}
	\item Leiten Sie Gleichung \eqref{eq:1} über den Zusammenhang zwischen Lorentz- und Zentripetalkraft her. Wie ist der Zusammenhang zwischen $e/m$ und den gemessenen Grössen $I,~U$ und $r$ ? Eliminieren Sie dabei die magnetische Induktion $B$ durch Verwendung des Kalibrierungsfaktors $K = B/I$ für die Helmholtzspulen! Wie kann bei konstantem Radius $r$ der Wert für $e/m$ graphisch ermittelt werden?
		\subitem Wie bereits oben klargestellt, kann man annehmen, dass $\vec{v}\perp\vec{B}$. Damit gilt:
		\begin{align*}
			\vec{v}\cdot\vec{B}&=0\\
			|\vec{v}\times\vec{B}|&=vB
		\end{align*}
		Somit gilt für die genannten Kräfte:
		\begin{align*}
			F_l&=F_z\\
			evB&=m\frac{v^2}{r}\\
			\frac{e}{m}&=\frac{v}{rB}\numberthis \label{eq:3}
		\end{align*}
		Mit Gleichung \eqref{eq:2} gilt:
		\begin{align*}
			\frac{v}{rB}&=\frac{v^2}{2U}\\
			v&=\frac{2U}{rB}
		\end{align*}
		Setzt man dies in \eqref{eq:3} ein, erhält man:
		\begin{displaymath}
			\frac{e}{m}=\frac{\frac{2U}{rB}}{rB}=\frac{2U}{r^2B^2}
		\end{displaymath}
		Nutzt man nun den Kalibrierungsfaktor $K=\frac{B}{I}$, so ergibt sich der Zusammenhang:
		\begin{displaymath}
			\frac{e}{m}=\frac{2U}{r^2K^2I^2}
		\end{displaymath}
		Hieraus sieht man sofort, dass e/m graphisch als Steigung eines $U-I^2-$Diagramms bzw. über den Achsenabschnitt eines $U-I-$Diagramms mit doppelt logarithmischer Skala bestimmt werden kann.
	\item Ein Elektron wird durch eine Spannung $U$ beschleunigt und unter einem Winkel $\alpha$ in ein Magnetfeld geschossen wie in Abb. \ref{fig:Abb5} gezeigt. Welche Bahn beschreibt das Elektron? Wie ändert sich der Bahnradius? Hinweis: Zerlegen Sie $\vec{v}$ in $v_\perp$ und $v_\parallel$ bzgl. $H$ und berechnen Sie daraus einerseits die Schraubenhöhe und andererseits auch die Umlaufzeit (Larmor-Frequenz).
	\begin{figure}[!h]
		\centering
		\includegraphics{emovement}
		\caption{Elektron im H-Feld}
		\label{fig:Abb5}
	\end{figure}
		\subitem Es gilt für die Geschwindigkeit:
		\begin{align*}
			v_\parallel&=v\cos\alpha\\
			v_\perp&=v\sin\alpha
		\end{align*}
		Hierbei leistet $v_\parallel$ keinen Beitrag zur Ablenkung durch die Lorentzkraft, da diese Komponente von $\vec{v}$ parallel zu $\vec{H}$ ist, das Kreuzprodukt verschwindet daher. Lediglich $v_\perp$ sorgt für eine Ablenkung des Elektrons, das im Folgenden eine Schraubenbahn beschreibt. Für die Schraubenhöhe $h$ gilt mit der Umlaufzeit $T$:
		\begin{displaymath}
			h= Tv_\parallel=Tv\cos\alpha
		\end{displaymath}
		Für den Radius der Kreisbahn ergibt sich aus dem Kräftegleichgewicht:
		\begin{align*}
			\mu_0eHv\sin\alpha&=m\frac{v^2\sin^2\alpha}{r}\\
			r&=\frac{mv\sin\alpha}{\mu_0eH}
		\end{align*}
		Daraus lässt sich nun die Umlaufzeit berechnen:
		\begin{displaymath}
			T=\frac{2\pi r}{v_\perp}=\frac{2\pi m}{\mu_0eH}
		\end{displaymath}
		Setz man dies in die obige Formel für die Ganghöhe ein, so kommt man schließlich zum Ergebnis:
		\begin{displaymath}
			h=\frac{2\pi mv\cos\alpha}{\mu_0eH}
		\end{displaymath}
	\item Wie kann man den Einfluss des Erdmagnetfelds auf die Elektronenbahn vermeiden?
		\subitem Am einfachsten kann dies bewerkstelligt werden, wenn man dafür sorgt, dass die Elektronen sich parallel zum Erdmagnetfeld bewegen. Ist dies nicht möglich, so kann Kompensation durch ein zum Erdmagnetfeld entgegengesetzt gleiches Feld erreicht werden.
\end{enumerate}
\section{Durchführung}
\subsection{Thomson-Röhre}
\begin{enumerate}
	\item Achten Sie darauf, dass während des gesamten Versuchs folgende Maxiamlwerte nicht überstiegen werden:
	\begin{itemize}
		\item Anodenspannung: 4 kV
		\item Spulenstrom: 0,9 A
	\end{itemize}
	\item Schließen Sie die Thomson-Röhre an das Hochspannungsnetzgerät an. Erhöhen Sie die Spannung, bis der Elektronenstrahl sichtbar ist (Raum muss abgedunkelt sein!!!).
	\item Legen Sie Spannung an die Spulen an und beobachten Sie den Strahlverlauf. Der Elektronenstrahlverlauf ist kreisförmig, die Ablenkung erfolgt in einer Ebene senkrecht zum elektromagnetischen Feld.
	\item Variieren Sie nun abwechselnd die Anodenspannung und den Spulenstrom, während Sie die andere Komponente konstant halten. Welche Auswirkung hat dies auf den Radius des Elektronenstrahls? Machen Sie sich die Zusammenhänge klar!
	\item Für die folgenden Schritte zur Bestimmung von $e/m$ stellen Sie nun einen Radius ein, der die aufgedruckte Skala schneidet.
	\item Bestimmung von $r$: Der Krümmungsradius $r$ des abgelenkten Elektronenstrahls lässt sich aus dem Austrittspunkt A mittels folgender Gleichung bestimmen:
	\begin{equation}
		r=\frac{80^2mm^2+e^2}{\sqrt{2}(80mm-e)}
	\end{equation}
	wobei sich $e$ direkt an der Skala ablesen lässt (siehe Abb \ref{fig:Abb6}).
	\begin{figure}[!h]
		\centering
		\includegraphics{emskizze}
		\caption{Bestimmung von r}
		\label{fig:Abb6}
	\end{figure}
	\item Bestimmung von $B$: Für die magnetische Flussdichte $B$ des Magnetfeldes bei Helmholtzgeometrie des Spulenpaars und dem Spulenstrom $I$ gilt:
	\begin{equation}
		B=\Big(\frac{4}{5}\Big)^\frac{3}{2}\cdot\frac{\mu_0\cdot n}{R}\cdot I=K\cdot I
	\end{equation}
	Der Kalibrier-Faktor für den angegebenen Aufbau ist $K=\SI{3,5}{\milli\tesla\per\ampere}$.
	\item Bestimmen Sie nun aus Spannung, Magnetfeld und Radius einen ersten Wert von $e/m$. Vergleichen Sie ihn mit dem akzeptierten Wert $e/m=\SI{1.76E11}{\ampere\second\per\kilogram}$. Falls Sie unvenünftig große Abweichungen feststellen, haben Sie wahrscheinlich irgendwo einen Fehler gemacht. Messen Sie erst weiter, wenn Sie diesen beseitigt haben.
	\item Nehmen Sie jetzt mehrere Wertepaare für $I$ und $U$ auf, wobei Sie darauf achten, den Radius $r$ konstant zu lassen.
	\item Machen Sie für Ihre Messungen eine genaue Fehlerbetrachtung, insbesondere unter Berücksichtigung der verschiedenen Wichtungen verschiedener Wertepaare. Beachten Sie dabei, dass auch die Bestimmung des Radius $r$ ungenau ist. Legen Sie diesen Fehler auf die Wertepaare $I$ und $U$ um. (Warum geht das?)
	\item Bestimmen Sie die spezifische Ladung $e/m$ eines Elektrons graphisch unter Berücksichtigung der Messfehler.
\end{enumerate}
\subsection{Doppelstrahlröhre}
\begin{enumerate}
	\item Achten Sie darauf, dass während des gesamten Versuchs folgende Maximalwerte nicht überstiegen werden:
	\begin{itemize}
		\item Plattenspannung: 45 V
		\item Spulenstrom: 0,4 A
	\end{itemize}
	\item Raumbeleuchtung abdunkeln, Heizspannung UF von 7 V einstellen und ca. 1 Minute warten bis sich die Temperatur der Heizung stabilisiert hat.
	\item Erhöhen Sie nun die Anodenspannung $U_A$ auf 100 V. Ohne anliegendes Magnetfeld erkennen Sie einen leuchtenden Strich.
	\item Stellen Sie den Spulenstrom $I_H$ so ein, dass ein geschlossener Kreis sichtbar wird. Beschreiben Sie, warum sich der anfänglich beobachtete Strich bei Erhöhung des Magnetfeldes dreht und zusammenzieht.
	\item Erhöhen Sie die Plattenspannung und beobachten Sie dessen Effekt auf den sichtbaren Elektronenkreis.
	\item Nachdem Sie sich mit dem Auswirkungen des Spulenstroms und der Plattenspannung auf die Elektronenbahn vertraut gemacht haben, variieren Sie nun beide Parameter um den Elektronenkreis möglichst genau an den Fluoreszenzschirm anzuschmiegen. Errechnen Sie unter Ausnutzung der Formel \eqref{eq:1} einen Wert für $e/m$. Schätzen Sie hierfür den Radius der Elektronenbahn ab (Tip: der Durchmesser des Glaskolbens beträgt 130 mm). Das Magnetfeld lässt sich durch die Helmholtz-Anordnung über folgende Beziehung bestimmen:
	\begin{displaymath}
		B^2=17,39\cdot 10^{-6}\cdot I_H^2
	\end{displaymath}
	\item Errechnen Sie $e/m$ für drei weitere Radien, wobei Sie die Plattenspannung konstant halten.
	\item Fehlerbetrachtung: Der kreisförmige Strahl ist sichtbar durch Photoemission. Warum ist der Fehler bei der Bestimmung von $e/m$ immer auf der negativen Seite?
\end{enumerate}
\chapter{Polarisation des Lichts}
Als Polarisation des Lichts bezeichnet man die (wohldefinierte) Richtung des elektrischen Feldvektors $\vec{E}$ im Bezug auf die Ausbreitungsrichtung $\vec{k}$ der Welle. Wie derartiges Licht erzeugt und nachgewiesen wird, sowie seine Eigenschaften, sollen in diesem Versuch erschlossen werden. 

Hierzu soll zunächst die mathematische Beschreibung von elektromagnetischen Wellen wiedergegeben werden. Dies geschieht durch die Maxwellgleichungen und die daraus abgeleiteten elektromagnetischen Wellengleichungen:
\begin{align}
	\label{eq:4}\divg{E}&=0\\
	\label{eq:6}\divg{B}&=0\\
	\rot{B}&=\epsilon_0\epsilon_r\pdiff{\vec{E}}{t}\\
	\label{eq:5}\rot{E}&=-\mu_0\pdiff{\vec{B}}{t}\\
	\Lap\vec{E}-\mu_0&\epsilon_0\epsilon_r\pddiff{\vec{E}}{t}=0\\
	\Lap\vec{B}-\mu_0&\epsilon_0\epsilon_r\pddiff{\vec{B}}{t}=0
\end{align}
Hieraus lässt sich mit der Lichtgeschwindigkeit $\sol=\frac{1}{\sqrt{\mu_0\epsilon_0}}$ leicht die wohlbekannte Lösung in Form einer ebenen Welle verifizieren.

Bei der Polarisation von Licht unterscheidet man hauptsächlich drei Fälle: linear, zirkular und elliptisch polarisiertes Licht. Vom ersten Fall spricht man, wenn die Schwingebene ortsfest ist, sich also bei der Projektion der Schwingung auf eine zur Ausbreitungsrichtung senkrechte Ebene das Bild einer geraden Linie ergibt. Man erhält diese Form der Polarisation bei der Überlagerung gleichamplitudiger, linear polarisierter Wellen mit Phasendifferenz $\Delta\varphi=k\cdot\pi,~ k\in\IZ$. Eine weitere Möglichkeit ist die Überlagerung gleich intensiver zirkular polarisierter Wellen. Die Richtung der Polarisation hängt dann von der Phasendifferenz der ursprünglichen Wellen ab.

Der Spezialfall der zirkular polarisierten Welle entsteht dann, wenn amplitudengleiche Wellen mit einer Phasendifferenz $\Delta\varphi=(k+\frac{1}{2})\pi,~k\in\IZ$ überlagert werden, bzw. bei der Überlagerung zirkular polarisierter Wellen, wobei die eine eine verschwindend geringe Amplitude besitzt. In jedem anderen Fall ergibt sich der allgemeine Fall einer elliptisch polarisierten Welle.

Im Folgenden sollen außerdem Methoden zur Erzeugung polarisierten Lichts diskutiert werden. Hierzu ist zunächst ein weiterer mathematischer Zusammenhang vonnöten, der die Intensität von Licht beschreibt, das einen idealen Polfilter durchläuft: Das Gesetz von Malus
\begin{equation}
	I(\theta)=I_0\cos^2\theta
\end{equation}
Im Fall unpolarisierten Lichtes muss noch über alle möglichen Winkel integriert werden:
\begin{equation}
	\avg{I}=\frac{1}{2\pi}\int_{0}^{2\pi}I_0\cos^2\theta\d\theta=\frac{1}{2}I_0
\end{equation}

Eine besonders in diesem Versuch, sowie zur Erzeugung sogenannter $\lambda/4-$ und $\lambda/2-$Plättchen wichtige Möglichkeit zur Lichtpolarisation ist die Doppelbrechung in Kristallen mit optischen Achsen. Dies passiert, wenn ein Kristall optisch anisotrop ist und mehrere ausgezeichnete Richtungen besitzt, entlang derer Licht gleicher Wellenlänge, aber unterschiedlicher Polarisation unterschiedlich stark gebrochen werden. So können für die sich aufspaltenden Lichtstrahlen bestimmte Gangunterschiede über die Dicke der Kristallplatte festgelegt werden. 

Ein ähnliches Prinzip liegt der optischen Aktivität organischer Moleküle zugrunde. Mittels bereits bekannter Werte für den spezifischen Drehwinkel eines bestimmten Stoffes kann so recht einfach seine Konzentration in einer Lösung bestimmt werden.
\section{Vorbereitung}
\begin{enumerate}
	\item Unter welchen Voraussetzungen sind die Maxwellgleichungen \eqref{eq:4}-\eqref{eq:5} gültig?
		\subitem Die Gleichungen gelten für ungeladene, stromfreie lineare Materialien mit skalaren Größen $\mu_r,\epsilon_r$. Im Fall anisotroper Materialien werden diese durch entsprechende Tensoren $\ten{\mu}_r,\ten{\epsilon}_r$ ersetzt.
	\item Unter welchen Bedingungen sind ebene Wellen Lösungen der Maxwellgleichungen?
		\subitem Die ebenen Wellen müssen die Dispersionsrelation $\omega=\sol k$, sowie die Transversalitätsbedingung $\skp{E}{k}=0$ bzw. $\skp{B}{k}=0$ erfüllen.
	\item Zeigen Sie, dass elektromagnetische Wellen Transversalwellen sind.
		\subitem Betrachte elektromagnetische Wellen der Form:
		\begin{align*}
			\vec{E}(\vec{r},t)&=\vec{E}_0\e{\imath(kz-\omega t)}\\
			\vec{B}(\vec{r},t)&=\vec{B}_0\e{\imath(kz-\omega t)}
		\end{align*}
		Im Zweifelsfall kann das zugrundeliegende Koordinatensystem immer so gelegt werden, dass $\vec{k}\cdot\uvec{z}=0$ gilt. Durch Einsetzen in die Maxwellgleichungen \eqref{eq:4} und \eqref{eq:6} ergibt sich folgende Bedingung:
		\begin{align*}
			\divg{E}&=E_{0,z}\e{\imath(kz-\omega t)}=0&&\Rightarrow E_{0,z}=0\\
			\divg{B}&=B_{0,z}\e{\imath(kz-\omega t)}=0&&\Rightarrow B_{0,z}=0
		\end{align*}
		Die Wellen stehen also senkrecht auf ihrer Ausbreitungsrichtung und sind somit transversal.
	\item Warum ist das natürliche Licht unpolarisiert?
		\subitem Natürliches Licht, wie z.B. Sonnenlicht ist als thermische Strahlung aus vielen Einzelwellen aufgebaut, deren Eigenschaften statistisch verteilt sind, wodurch keine klare Polarisation vorgegeben ist.
	\item Nennen Sie die möglichen Polarisationszustände von Licht. Wie kann man diese mathematisch darstellen?
		\subitem Analog zur Einführung kann Licht linear, zirkular oder allgemein elliptisch polarisiert bzw. ganz unpolarisiert sein. Auch die mathematischen Grundlagen für die ersten beiden Polarisationsarten wurden bereits genannt. Bei elliptisch polarisiertem Licht liegt eine Überlagerung zweier Wellen mit im Allgemeinen unterschiedlicher Amplitude und Phase, aber gleicher Ausbreitungsrichtung vor, während bei unpolarisiertem Licht eine Überlagerung sehr vieler Wellen mit unterschiedlichen Eigenschaften wie Amplitude, Phase, Wellenlänge, Polarisation und Ausbreitungsrichtung möglich ist. 
		
		Als Überlagerung bezeichnet man hierbei jeweils die Summe einzelner Teilwellen.
	\item Wie kann man einen Linearpolarisator von einer Verzögerungsplatte unterscheiden?
		\subitem Man bestrahle das zu untersuchende Objekt mit linear polarisiertem Licht. Der Polarisator wird unter Drehung eine veränderliche Lichtintensität transmittieren, bei einem bestimmten Winkel, genauer $\pm\frac{\pi}{2}$ zur Polarisationsrichtung des Lichtes wird nach dem Gesetz von Malus kein Licht mehr durchgelassen. Bei einer Verzögerungsplatte hingegen wird lediglich die Polarisation des transmittierten Lichts geändert, z.B. wird das Licht gedreht oder zirkulare Polarisation erzeugt. Dies kann durch nachgeschaltete Polfilter oder Verzögerungsplatten mit bekannten Eigenschaften nachgewiesen werden.
	\item Welche Eigenschaften haben Linearpolarisator, $\lambda/2-$ und $\lambda/4-$Plättchen? Erklären Sie, welche Polarisationszustände man aus lin. pol. Licht mit einem $\lambda/2-$ bzw- $\lambda/4-$Plättchen erzeugen kann. Welchen Einfluss hat dabei die optische Achse (Vorzugsrichtung) des Plättchens?
		\subitem Ein Linearpolarisator filtert einfallende Strahlung so, dass nur linear polarisiertes Licht mit festgelegter Polarisationsrichtung durchgelassen wird. Bei Verzögerungsplatten hingegen werden einfallende Lichtstrahlen formal "`aufgespalten"' und die unterschiedlichen optischen Weglängen im anisotropen Kristall ausgenutzt, um festgelegte Gangunterschiede zwischen den ausfallenden Strahlen zu erzeugen. Hierbei erzielt ein $\lambda/2-$Plättchen einen Phasenversatz von $\pi$, ein $\lambda/4-$Plättchen entsprechend um $\frac{\pi}{2}$. Zudem sind Polarisatoren im Allgemeinen von der Farbe des Lichts unabhängig, während Verzögerungsplatten namensgetreu nur für eine bestimmte Wellenlänge wirksam sind.
		
		Mit ersterem wird linear polarisiertes Licht erneut linear polarisiert ausgegeben, jedoch um den Winkel $2\alpha$ gedreht, wenn $\alpha$ der Winkel zwischen Polarisationsrichtung und optischer Achse des Kristalls ist.
		
		Zweiteres erzeugt aus linear polarisiertem Licht i.A. elliptisch polarisiertes. Für $\alpha=\frac{\pi}{4}$ bzw. $\alpha=0$ mit $\alpha$ wie oben erhält man zirkular bzw. linear polarisiertes Licht. In jedem Fall beschreibt das austretende Licht in einer zur Ausbreitung senkrechten Ebene eine sogenannte Lissajous-Figur.
	\item Wie dick muss ein $\lambda/2-$ oder $\lambda/4-$Plättchen aus Glimmer sein, wenn die Wellenlänge des einfallenden Lichtes $\SI{589}{\nano\meter}$ ist?
		\subitem Es gilt für den Phasenversatz folgende Identität:
		\begin{displaymath}
			\Delta\varphi=\frac{2\pi}{\lambda}d(n_\gamma-n_\beta)
		\end{displaymath}
		Man benutze die Brechindizes für Glimmer: $n_\gamma=1.5993,~n_\beta=1.5944$\\
		Durch Umstellung erhält man für $d$:
		\begin{displaymath}
			d=\frac{\Delta\varphi\lambda}{2\pi(n_\gamma-n_\beta)}
		\end{displaymath}
		Für Phasenversätze $\Delta\varphi=k\pi$ bzw. $\Delta\varphi=(k+\frac{1}{2})\pi$ erhält man für die Dicke der jeweiligen Plättchen:
		\begin{align*}
			d_{\lambda/2}&=k\cdot\SI{6.01e-5}{\meter}\\
			d_{\lambda/4}&=(k+\frac{1}{2})\cdot\SI{6.01e-5}{\meter}
		\end{align*}
	\item Wann nennt man optische Medien isotrop bzw. anisotrop? Geben Sie jeweils zwei Beispiele an.
		\subitem Isotrope Medien haben in jeder Richtung das gleiche Brechverhalten bzw. den gleichen Brechungsindex. Beispiele sind optisch inaktive Fluide wie Luft oder Wasser.
		
		Anisotrope Medien hingegen besitzen genau diese Eigenschaft nicht, Lichtbrechung ist also von der Durchstrahl- oder Polarisationsrichtung des Lichtes abhängig. Dieses Verhalten ist etwa bei Glimmer oder Kalkspat zu betrachten.
	\item Auf ein $\lambda/4-$Plättchen aus Quarz fällt Licht einer Natriumlampe ($\lambda=\SI{589}{\nano\meter}$). Wie dick ist die Quarzplatte? Welche Frequenz und Wellenlänge haben ordentlicher und außerordentlicher Strahl innerhalb des Kristalls?
		\subitem Man benutze statt der (unbekannten) Brechungsindizes die bekannte Doppelbrechung von Quarz, die von der Wellenlänge zumindest näherungsweise unabhängig ist: $\delta=(n_e-n_o)=0.0091$\\
		Damit ergibt sich für die Dicke der Platte:
		\begin{displaymath}
			d=(k+\frac{1}{2})\frac{\lambda}{2\delta}=(k+\frac{1}{2})\cdot\SI{6.47e-5}{\meter}
		\end{displaymath}
		Die Frequenz des Lichtes ist vom Medium unabhängig:
		\begin{displaymath}
			\nu=\frac{c}{\lambda}=\SI{5.09e14}{\tera\hertz}
		\end{displaymath}
		Für die Wellenlängen gilt:
		\begin{align*}
			\lambda_o&=\frac{\lambda}{n_o}\\
			\lambda_e&=\frac{\lambda}{n_e}
		\end{align*}
		Da die Brechungsindizes bei der angegebenen Wellenlänge unbekannt sind, werden im Folgenden die Brechzahlen für $\lambda=\SI{590}{\nano\meter}$ verwendet. Diese sind: $n_o=1.544,~n_e=1.553$. Damit ergibt sich:
		\begin{align*}
			\lambda_o&=\SI{381.4}{\nano\meter}\\
			\lambda_e&=\SI{379.3}{\nano\meter}
		\end{align*}
	\item Die Durchlassrichtung von zwei hintereinander stehenden, idealen Polarisatoren sind um den Winkel $\alpha_1=\SI{30}{\degree}$ gegeneinander verdreht. Auf die Anordnung fällt Licht, dessen Schwingungsrichtung den Winkel $\alpha_2=\SI{15}{\degree}$ mit der Durchlassrichtung des ersten Polarisators bildet. Wie groß ist der Transmissionsgrad dieser Anordnung?
		\subitem \begin{displaymath}
			\frac{I}{I_0}=\cos^2\alpha_2\cos^2\alpha_1=0.70
		\end{displaymath}
	\item Was versteht man unter optischer Aktivität?
		\subitem Unter optischer Aktivität versteht man die Eigenschaft organischer Materialien wie Zucker oder Milchsäure, die Schwingrichtung linear polarisierten Lichtes um einen bestimmten Winkel zu drehen.
	\item Beschreiben Sie den Strahlengang und die Funktionsweise des Nicol'schen Prismas.
		\subitem \begin{figure}[!h]
			\centering
			\includegraphics{nicol}
			\caption{Strahlengang im Nicol'schen Prisma}
			\label{fig:Abb7}
		\end{figure}
		Ein negativ einachsiger ($n_o>n_e$) Kristall wird schräg zur optischen Achse durchgeschnitten und mit einem durchsichtigen Kleber wieder zusammengeklebt, sodass $n_o>n_K>n_e$ erfüllt ist. Aufgrund der unterschiedlichen Einfallswinkel der Strahlen auf die Kristall-Kleber-Grenzfläche kann so bewirkt werden, dass der Einfallswinkel des ordentlichen Strahls größer ist, als der Totalreflexionswinkel. Es tritt daher nur der außerordentliche Strahl aus, der parallel zur Einfallsebene und somit linear polarisiert ist.
	\item Was versteht man unter Spannungsdoppelbrechung?
		\subitem Unter Spannungsdoppelbrechung versteht man die Eigenschaft von Materialen, unter mechanischer Zugspannung Doppelbrechung aufzuweisen, also in verschiedenen Richtungen verschiedene Brechungsindizes zu besitzen.
	\item Wie kann man Licht eines bestimmten Polarisationstyps erzeugen und nachweisen?
		\subitem Die Erzeugung linear polarisierten Lichts aus unpolarisiertem geschieht zum Beispiel durch bereits beschriebene Mechanismen der Doppelbrechung. Daraus lässt sich mittels eines $\lambda/4-$Plättchens zirkular polarisiertes Licht erzeugen. Der Nachweis dieser Polarisationen geschieht durch Linearpolarisatoren im ersten Fall bzw. durch die Kombination eines $\lambda/4-$Plättchens und eines Linearpolarisators im zweiten Fall.
\end{enumerate}
\section{Durchführung}
\subsection{Doppelbrechung}
Legen Sie den Kalkspat-Kristall auf Millimeterpapier, auf das Sie zuvor ein farbiges Kreuz gezeichnet haben. Bestimmen Sie nun durch Drehen des Kristalls die Strahlverschiebung $s$. Fällt das Licht senkrecht auf die Oberfläche des Kristalls, so kann aus der Stahlverschiebung und der Kristalldicke $d$ der Ablenkwinkel berechnet werden (siehe Abb.  \ref{fig:Abb8}).
\begin{figure}[!h]
	\centering
	\includegraphics{doppelbrechung}
	\caption{Ordentlicher und außerordentlicher Strahl im Kalkspat}
	\label{fig:Abb8}
\end{figure} 
Die Formel zur Berechnung des Ablenkwinkels $\Delta\vartheta$ können Sie der Abb. \ref{fig:Abb8} entnehmen. Vergleichen Sie die gemessene Strahlverschiebung $s$ mit dem theoretischen Wert der Strahlverschiebung des Kalkspats. Die Winkel $\vartheta_k$ und $\vartheta_s$ sind wie folgt verknüpft:
\begin{equation}
	\label{eq:7}\frac{\tan\vartheta_s}{\tan\vartheta_k}=\frac{n_o^2}{n_e^2}
\end{equation}
Die optische Achse tritt unter dem Winkel $\alpha=\SI{45.49}{\degree}$ aus der Kristalloberfläche aus. Setzen Sie $\vartheta_s$ aus Gleichung \eqref{eq:7} in ihre Gleichung für die Strahlverschiebung ein. Mit $\vartheta_k=\SI{90}{\degree}-\alpha$ können Sie den theoretischen Wert der Strahlverschiebung berechnen.
\subsection{Gesetz von Malus}\label{GvM}
Machen Sie sich zunächst mit dem Polarisator vertraut. Überprüfen Sie die Angabe der Durchlassrichtung (E-Vektor). Beobachten Sie die Polarisationseigenschaften von Gegenständen in ihrer Umgebung sowie der Atmosphäre und der Wolken und notieren Sie diese. Wie schwingt der E-Vektor des beobachteten Lichtes?

Für die Messungen stellen Sie einen Polarisator und einen Analysator mit Winkelmessvorrichtung in den Strahlengang (vgl Abb. pl.2). Machen Sie sich zunächst klar, wann deren Durchlassrichtungen parallel bzw. senkrecht zueinander stehen. Messen Sie nun die Winkelintensitätsverteilung von polarisiertem Licht, wobei Sie den Winkel zwischen Polarisator und Analysator kontinuierlich ändern. Überprüfen Sie das Gesetz von Malus durch eine geeignete graphische Darstellung ihrer Messwerte. Tragen Sie dann ihre Messwerte in Polarkoordinatenpapier ein. Was sagt die Graphik aus?
\subsection{$\lambda/4-$Plättchen}\label{4tel}
Stellen Sie nun ein $\lambda/4-$Plättchen zwischen den Polarisator und den Analysator in den Strahlengang. Orientieren Sie das Plättchen mit Hilfe der Polarisatoren. Erzeugen Sie nun elliptisch polarisiertes Licht. Überlegen Sie sich wie das $\lambda/4-$Plättchen eingestellt werden muss! Messen Sie die Intensitätsverteilung wie unter \ref{GvM} und tragen Sie ihre Messwerte in Polarkoordinatenpapier ein. Stellen Sie das $\lambda/4-$Plättchen nun so in den Strahlengang, dass es zirkular polarisiertes Licht erzeugt. Messen Sie erneut die Intensitätsverteilung wie unter \ref{GvM} und tragen Sie ihre Messwerte in Polarkoordinatenpapier ein.
\subsection{$\lambda/2-$Plättchen}\label{halb}
Tauschen Sie das $\lambda/4-$Plättchen durch ein $\lambda/2-$Plättchen aus und orientieren Sie dieses im Strahlengang. Drehen Sie es anschließend um $\SI{20}{\degree}$ zur eingestellten Vorzugsrichtung. Messen Sie die Intensitätsverteilung und tragen Sie Ihre Messwerte in Polarkoordinatenpapier ein. Was schließen Sie aus dem Ergebnis?
\subsection{selbstgebaut...}
Wiederholen Sie die Aufgaben \ref{4tel} und \ref{halb} mit "`selbstgebauten"' Plättchen.
\subsection{Beobachtungen}
Betrachten Sie transparente Medien unter gekreuzten Polarisatoren und schreiben Sie Ihre Beobachtungen nieder.

\end{document}

