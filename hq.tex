\chapter{Bestimmung des Planck'schen Wirkungsquantums}
In diesem Versuch soll das Planck'sche Wirkungsquantum $\planck$ bestimmt werden. Diese Größe ist neben der Lichtgeschwindigkeit $\sol$ und der Gravitationskonstanten $\G$ die dritte fundamentale Naturkonstante der Physik. Sie beschreibt für schwingfähige Systeme das konstante Verhältnis aus der kleinstmöglichen Energieänderung und der Schwingungsfrequenz. Daraus folgt insbesondere, dass solche Systeme nur ganzzahlige Vielfache des sog. Schwingungsquants $\Delta E=\planck\nu$ aufnehmen können. Auch der Drehimpuls eines Systems kann sich nur um ganzzahlige Vielfache von $\hbar\equiv\frac{\planck}{2\pi}$ ändern. Dies sind jedoch nur einige der wichtigen Zusammenhänge, in denen die Planckkonstante eine wichtige Rolle spielt.

Die Bestimmung dieser fundamentalen Konstanten soll in diesem Versuch durch Messungen an Leuchtdioden, insbesondere der Feststellung von deren Wellenlängen über die Gleichung 
\begin{equation}
	E=\planck\nu
\end{equation}
bestimmt werden. Zwischen der Wellenlänge $\lambda$ und der Frequenz $\nu$ besteht dabei der bekannte Zusammenhang: $\sol=\lambda\nu$.

Da solche LEDs Halbleiter sind, muss in diesem Versuch auch eine gewisse Vertrautheit mit Grundlagen der Halbleiterphysik gegeben sein. Für diesen Versuch ist insbesondere das Resultat wichtig, dass die Wellenlänge des abgestrahlten Lichts nur von der sogenannten Gap-Energie $E_{\mathtt{gap}}$ abhängt, also von der Energie, die frei wird, wenn ein Elektron vom Leitungsband, der Energiezone, in der sich frei bewegliche und damit leitungsfähige Ladungsträger befinden, in das energetisch tieferliegende Valenzband übergeht. Diese Energie ist eine Materialkonstante, womit klar wird, dass die Farbe der LED nur von den verwendeten Halbleitern abhängt:
\begin{equation}
	\lambda=\frac{\planck\cdot\sol}{E_{\mathrm{gap}}}
\end{equation}
\section{Vorbereitung}
